\documentclass[thesis.tex]{subfiles}
\begin{document}
\onehalfspacing
% Change math display spacing here, so spacing commands don't reset it.
\abovedisplayskip=3pt
\belowdisplayskip=3pt
\abovedisplayshortskip=2pt
\belowdisplayshortskip=2pt

  \chapter{Distributions}
  \label{ch:dist}
  
  With a firm grasp of the topology of \S, we now turn back to topological issues.
  Recall from \cref{sec:tempdist} that continuous linear dual of \S is the space of \emph{tempered distributions}:
  \begin{align*}
    \S^* = \{f:\S\rightarrow\C\,\vert\,f\text{ is continuous and linear}\}
  \end{align*}
  This definition is now fully sensible, as we know what it means to be a continuous functional on \S.
  Recall the mapping taking a test function $\varphi$ to the integration-against-$\varphi$ functional on \S:
  \begin{align*}
    \varphi \longmapsto \left( \phi \mapsto \int_{x\in\R} \varphi(x) \phi(x)\,dx \right)
  \end{align*}
  We hope that this map takes elements of \S into elements of $\S^*$, as we might expect it to.
  \begin{thm}
    The integration-against functional is continuous and linear.
    Thus, the mapping above gives an inclusion $\S\rightarrow\S^*$.
  \end{thm}
  \begin{proof}
    Linearity follows trivially from the definition.
    Thus we need only show that small inputs give small outputs, and continuity everywhere will follow from linearity.
    Fix some $\varphi\in\S$ and let $\{\phi_i\}$ be a sequence in \S approaching the zero function (which is trivially Schwartz).
    See that
    \begin{align*}
      |\langle\varphi,\phi_i\rangle| 
      &= \left|\int_{x\in\R} \varphi(x)\phi_i(x)\,dx\right|
      \\&\le \int_{x\in\R}| \varphi(x)\phi_i(x)|\,dx
      \\&= \int_{x\in\R}|\varphi(x)|\cdot|\phi_i(x)|\,dx
      \\&\le \int_{x\in\R}|\varphi(x)|\cdot|\phi_i|_{C^k_\ell}\,dx
      \\&= |\phi_i|_{C^k_\ell}\cdot\int_{x\in\R}|\varphi(x)|\,dx
    \end{align*}
    for any $k$ and $\ell$, by definition of the $C^k_\ell$ norm.
    Because $\{\phi_i\}$ approaches zero in \S, we see that $\{|\phi_i|_{C^k_\ell}\}$ also approaches zero.
    Thus the sequence $\{\langle\varphi,\phi_i\rangle\}$ approaches zero in \C, so the functional is continuous.
  \end{proof}

  The continuity of the integration-against functional shows that each Schwartz function corresponds naturally to a tempered distribution.
  In \cref{sec:tempdist} we saw that this inclusion allowed us to define the Fourier transform for any $f\in\S^*$ by the formula
  \begin{align*}
    \langle\F f, \varphi\rangle &= \langle f, \F\varphi\rangle
  \end{align*}
  for $\varphi\in\S$.
  
  We may define the derivative of a tempered distribution similarly.
  For Schwartz functions $f$ and $\varphi$, apply integration by parts:
  \begin{align*}
    \langle f', \varphi\rangle
    &= \int_{x\in\R} f'(x)\varphi(x)\,dx
    \\&= [f(x)\varphi(x)]_{-\infty}^\infty -\int_{x\in\R} f(x)\varphi'(x)\,dx
    \\&= -\int_{x\in\R} f(x)\varphi'(x)\,dx
    \\&= -\langle f,\varphi'\rangle
  \end{align*}
  \begin{defn}
    For a tempered distribution $u$, the distributional derivative of $u$ (denoted as usual by $u'$) is given by
    \begin{align*}
      \langle u', \varphi\rangle = -\langle u, \varphi'\rangle
    \end{align*}
    for any Schwartz function $\varphi$.
    When $u$ corresponds to a Schwartz function $f$, the distribution $u'$ visibly corresponds to the derivative $f'\in\S$ by the calculation above. 
  \end{defn}

  \todo{introduce weak-* topology}
  \begin{claim}
    The differentiation map $D:\S^*\rightarrow\S^*$ is continuous.
  \end{claim}
  \begin{proof}
    
  \end{proof}


\end{document}
