\documentclass[thesis.tex]{subfiles}
\begin{document}
\onehalfspacing
% Change math display spacing here, so spacing commands don't reset it.
\abovedisplayskip=3pt
\belowdisplayskip=3pt
\abovedisplayshortskip=2pt
\belowdisplayshortskip=2pt

  \setcounter{chapter}{2}
  \setcounter{thm}{0}
  \chapter{Distributions}
  \label{ch:dist}
  
  With a firm grasp of the topology of \S, we now turn back to topological issues.
  Recall from \cref{sec:tempdist} that continuous linear dual of \S is the space of \emph{tempered distributions}:
  \begin{align*}
    \S^* = \{f:\S\rightarrow\C\,\vert\,f\text{ is linear and continuous}\}
  \end{align*}
  This definition is now fully sensible, as we know what it means to be a continuous functional on \S.
  Recall the mapping taking a Schwartz function $\varphi$ to the integration-against-$\varphi$ functional on \S:
  \begin{align*}
    \varphi \longmapsto \left( \phi \mapsto \int_{x\in\R} \varphi(x) \phi(x)\,dx \right)
  \end{align*}
  We hope that this map takes elements of \S into elements of $\S^*$, as we might expect it to.
  \begin{thm}
    \label{thm:intagainstiscont}
    For any $\varphi\in\S$ the integration-against-$\varphi$ functional on \S is linear and continuous.
    Thus the mapping above gives an inclusion of sets $\S\rightarrow\S^*$.
  \end{thm}
  \begin{proof}
    Linearity follows trivially from the definition.
    Thus we need only show that small inputs give small outputs, and continuity everywhere will follow from linearity.
    Let $\{\phi_i\}$ be a sequence in \S approaching the zero function.
    Then
    \begin{align*}
      \left|\int_{x\in\R} \varphi(x)\phi_i(x)\,dx\right|
      \le |\varphi|_{L^1}\cdot |\phi_i|_\Coo
    \end{align*}
    Because $\{\phi_i\}$ approaches zero in \S, $\{|\phi_i|_\Coo\}$ also approaches zero.
    Thus the sequence $\{\langle \varphi,\phi_i\rangle\}$ approaches zero in \C, so the functional is continuous.
  \end{proof}
  The proof above trivially shows that any $L^1$-integrable function $f$ gives rise to a tempered distribution by integration-against, even if $f$ fails to be Schwartz.
  However, we wonder if there is some even larger family of classical functions that has such an inclusion into \SS.
  Recall that \S is closed under multiplication by powers of $x$.
  In fact, for a sequence $\{\phi_i\}\rightarrow 0$ in \S and any $\ell\in\Z_{>0}$, see that $\{(1+x^2)^\ell\phi_i(x)\}\rightarrow 0$ for any fixed $x\in\R$, and so by \cref{thm:sevalcts} $\{(1+x^2)^\ell\phi_i\}\rightarrow 0$.
  Thus, pretending that some function $f$ does give rise to a distribution $u_f$, write for any $\ell\in\Z_{>0}$:
  \begin{align*}
    |\langle u_f,\phi_i\rangle| 
    = \int_{x\in\R} \left( (1+x^2)^\ell\phi_i(x) \right) \cdot \left( (1+x^2)^{-\ell} f(x) \right) \,dx
  \end{align*}
  which approaches zero as $i$ grows, provided the integral is finite.
  Thus a classical function $f$ gives rise to a tempered distribution $u_f$ if there is an $\ell$ so that $(1+x^2)^{-\ell}f$ times any Schwartz function is $L^1$-integrable.

  \begin{defn}
    A continuous function $f$ is called a \emph{slowly increasing function} if there exists some $\ell\in\Z_{\ge0}$ so that
    \begin{align*}
      \lim_{|x|\rightarrow\infty} \left| \frac{f(x)}{(1+x^2)^{\ell}} \right| \,dx < \infty \text{.}
    \end{align*}
    If $f$ is infinitely differentiable and all of its derivatives (including the zeroth) is slowly increasing, then $f$ is called a \emph{fairly good function}.
  \end{defn}
  See that any slowly increasing function $f$ gives rise to a tempered distribution $u_f$ by integration-against-$f$ because the product of slowly increasing and Schwartz is \li:
  \begin{lemma}
    \label{lemma:sitimesschwartzli}
    Suppose that $f$ is a slowly increasing function.
    Then for any $g\in\S$, the product $fg$ is \li.
  \end{lemma}
  \begin{proof}
    Because $f$ is slowly increasing there is some $\ell$ so that the function $(1+x^2)^\ell f$ is bounded.
    Thus there is some $B$ so that $|(1+x^2)^{-\ell} f(x)|<B$ for all $x\in\R$.
    So calculate
    \begin{align*}
      \int_{x\in\R} |(1+x^2)^\ell g(x)|\cdot|(1+x^2)^{-\ell} f(x)|\,dx
      &\le B\cdot \int_{x\in\R} |(1+x^2)^\ell g(x)|\,dx
    \end{align*}
    which is finite because $g$ is Schwartz.
  \end{proof}
  \begin{lemma}
    Suppose that $f$ is a fairly good function.
    Then for any $g\in\S$, the product $fg$ is Schwartz.
  \end{lemma}
  \begin{proof}
    By the product rule, the product of smooth functions is smooth, thus we need only show that the product (and its derivatives) decay quickly.
    For any $\ell>0$ see that $(1+x^2)^\ell (fg)(x) = \left(f\cdot((1+x^2)^\ell g)\right)(x)$ which is integrable by \cref{lemma:sitimesschwartzli}.
    Thus $(1+x^2)^\ell fg$ vanishes at infinity for any $\ell$.
    By the product rule, any derivative $fg^{(i)}$ is a sum of fairly good times Schwartz, and so each term vanishes independently at infinity.
    Thus $fg$ is Schwartz.
  \end{proof}

  Notably, any polynomial is a fairly good function, and so gives rise to a tempered distribution.
  However, the exponential function $e^x$ is not slowly increasing, as it grows faster than any power of $x$ for large $x$.
%  We wonder if the converse of this is also true, so the only functions giving rise to distributions are slowly increasing functions.
%
%  \begin{thm}
%    If a distribution $u$ is given by integration against a classical function $f$, then $f$ is slowly increasing
%  \end{thm}
%  \begin{proof}
%    For any Schwartz function $\varphi$, the quantity
%    \begin{align*}
%      | \langle u, \varphi\rangle |
%      &= \left| \int_{x\in\R} f(x)\varphi(x)\,dx \right|
%      \call C
%    \end{align*}
%    is finite.
%  \end{proof}

  \section{Adjoint Identities and Topology on \SS}
    The continuity of the integration-against functional shows that each Schwartz function gives rise to a tempered distribution.
    In \cref{sec:tempdist} we saw that this inclusion allowed us to define the Fourier transform on \SS by the formula
    \begin{align*}
      \langle\F u, \varphi\rangle &= \langle u, \F\varphi\rangle
      \qquad u\in\SS, \varphi\in\S \text{.}
    \end{align*}
    
    We may define the derivative of a tempered distribution similarly.
    For Schwartz functions $f$ and $\varphi$, apply integration by parts and observe that because $f\varphi$ vanishes at infinity:
    \begin{align*}
      \langle f', \varphi\rangle
      &= \int_{x\in\R} f'(x)\varphi(x)\,dx
      = -\int_{x\in\R} f(x)\varphi'(x)\,dx
      = -\langle f,\varphi'\rangle\text{.}
    \end{align*}
    \begin{defn}
      For a tempered distribution $u$, the distributional derivative of $u$ (denoted as usual by $u'$) is given by
      \begin{align*}
        \langle u', \varphi\rangle = -\langle u, \varphi'\rangle
        \qquad u\in\SS,\varphi\in\S \text{.}
      \end{align*}
      When $u$ is given by (integration against) a Schwartz function $f$, the distribution $u'$ is given by the derivative $f'\in\S$ by the calculation above. 
    \end{defn}
    
    As with the Fourier Transform, the definition of the distributional derivative extends good properties of \S to all of $\S^*$.
    It is immediately obvious that any tempered distribution has infinitely many derivatives which are all themselves tempered distributions.
    We might also expect that, as on $\S$, the differentiation map is continuous on $\S^*$, but this involves topology on $S^*$, which we have not yet developed.
    We thus give the space of tempered distributions the so-called ``initial topology'', which allows us to carry topology from a space to its dual.
    \begin{defn}
      Let $X^*$ be the continuous linear dual of a topological space $X$.
      The initial topology on $X^*$ with respect to its linear dual, also called the \emph{weak-* topology} is the coarsest topology ensuring that for all $\varphi\in X^*$ the evaluation functional
      \begin{gather*}
        u_\varphi: X^*\rightarrow\C, \qquad f\mapsto\langle\varphi,f\rangle
      \end{gather*}
      is continuous.
    \end{defn}

    \begin{claim}
      The differentiation map $D:\SS\rightarrow\SS$ is continuous under the \ws.
    \end{claim}
    \begin{proof}
      Consider a sequence of tempered distributions $\{u_i\}$ that converges to zero.
      By definition of the \ws, we have for any Schwartz function $\varphi$, the sequence $\{\langle u_i,\varphi\rangle\}$ converges to zero by continuity of the evaluation-at-$\varphi$ functional.
      Thus the sequence $ \{\langle Du_i, \varphi\rangle\} = \{-\langle u_i, \varphi'\rangle\} \rightarrow 0 $ and so $\{Du_i\}\rightarrow 0$ in the \ws.
    \end{proof}

    \begin{thm}
      The inclusion $\S\rightarrow\SS$ given by $\varphi\longmapsto\left( \phi\mapsto \int_{x\in\R} \varphi(x)\phi(x)\,dx \right)$ is linear and continuous, and thus gives an inclusion of topological vectorspaces.
    \end{thm}
    \begin{proof}
      Linearity is immediate from the definition and continuity follows from the same inequality as \cref{thm:intagainstiscont} with $\varphi$ and $\phi$ exchanged.
    \end{proof}

  \section{Approximation by Schwartz Functions}
    A simple distribution is \emph{Dirac delta}:
    \begin{gather*}
      \langle\delta,\varphi\rangle = \varphi(0)
      \qquad\text{for any }\varphi\in\S
    \end{gather*}
    Because $\delta$ is trivially linear, and continuous by \cref{thm:sevalcts}, $\delta\in\SS$.
    In fact, $\delta$ is not a classical function, and so is a ``true distribution'' in that it does not arise from a function with pointwise values.
    If $\delta$ were a classical function of any sort, we would have
    \begin{gather*}
      \delta(x)=0\text{ for }x\ne0\\
      \int_{x\in\R} \delta(x)\,dx = 1
    \end{gather*}
    but for any $z=\delta(0)\in\C$ the integral must be zero.
    Instead we think of $\delta$ as an infinitesimally narrow spike with area 1.
    This idea allow us to see that $\delta$ may be approximated by Schwartz functions, in the sense that there is a sequence $\{\phi_n\}$ in \S so that for any $\varphi\in\S$
    \begin{align*}
      \langle \phi_n, \varphi\rangle \underset{n}{\longrightarrow} \varphi(0)\text{.}
    \end{align*}
    Because $\{\langle\phi_n,\varphi\rangle\}\rightarrow\langle\delta,\varphi\rangle$ for arbitrary $\varphi\in\S$, by definition $\{\phi_n\}\rightarrow\delta$ in the \ws.

    \begin{defn}[Approximate Identity]
      A sequence of Schwartz functions $\{\phi_n\}$ is called an \emph{approximate identity} whenever
      \begin{itemize}
        \item $\phi_n(x)$ is non-negative and real for all $n$ and $x\in\R$;
        \item $\int_{x\in\R}\phi_n(x)\,dx=1$ for all $n$;
        \item for any $\rho>0$ there exists some $N$ so that for all $n>N$ 
          \begin{align*}
            \phi_n(x) = 0 \text{ whenever } |x|\ge\rho \text{.}
          \end{align*}
      \end{itemize}
    \end{defn}
    The idea of an approximate identity is to construct a sequence of functions with constant ``mass'' 1 that clusters tightly around zero as $n$ increases.
    Intuitively the limit of such a function is exactly $\delta$: a unit point-mass gathered at zero.

    \begin{thm}[Approximate Identity Theorem]
      \label{thm:approxid}
      Given an approximate identity $\{\phi_n\}$, $\lim_n\langle \phi_n,\varphi\rangle = \varphi(0)$ for any $\varphi\in\S$.
    \end{thm}
    \begin{proof}
      For an arbitrary $\varphi\in\S$ we hope the quantity
      \begin{align*}
        \langle\phi_n,\varphi\rangle 
        &=\int_{x\in\R} \phi_n(x)\varphi(x)\,dx
      \end{align*}
      approaches $\varphi(0)$ as $n$ grows.
      See that
      \begin{align*}
        |\langle\phi_n,\varphi\rangle - \varphi(0)|
        &=\left| \int_{x\in\R} \phi_n(x)(\varphi(x)-\varphi(0))\,dx \right|
        \\&\le \int_{x\in\R} \phi_n(x)|\varphi(x)-\varphi(0)|\,dx
%        &= \int_{x\in\R} \phi_n(x)\varphi(x)\,dx - \int_{x\in\R} \phi_n(x)\varphi(0)\,dx
%        \\&= \int_{x\in\R} \phi_n(x)\varphi(x)\,dx - \varphi(0)
      \end{align*}
      which is arbitrarily small for large $n$ because $\varphi$ is continuous and $\phi_n$ kills everything outside a small neighborhood of 0.
      Thus we may suppose WLOG that $\varphi(0)=0$.

      First, for some $\rho>0$, see that there is some $N$ so that for $n>N$
      \begin{align*}
        \left| \int_{|x|\ge\rho} \phi_n(x)\varphi(x)\,dx \right|
        = 0
      \end{align*}
      by the definition of approximate identity.

      It remains to show that the integral on $(-\rho,\rho)$ converges to the desired value.
      By the continuity of $\varphi$, there is some $\varepsilon_\rho$ so that $|x|<\rho$ implies $|\varphi(x)|<\varepsilon_\rho$.
      Hence
      \begin{align*}
        \left| \int_{|x|<\rho} \phi_n(x)\varphi(x)\,dx \right|
        &< \varepsilon_\rho\cdot \left| \int_{|x|<\rho} \phi_n(x)\,dx \right|
        \le \varepsilon_\rho
      \end{align*}
      because $\int_{|x|<\rho} \phi_n(x)\,dx \le 1$ for all $n$.

      Thus, there is some $N$ so that for any $n>N$
      \begin{align*}
        \left| \int_{x\in\R} \phi_n(x)\varphi(x)\,dx \right|
        &\le \left| \int_{|x|\ge\rho} \phi_n(x)\varphi(x)\,dx \right|
        + \left| \int_{|x|<\rho} \phi_n(x)\varphi(x)\,dx\right|
        < 0+\varepsilon_\rho \text{.}
      \end{align*}
      By the continuity of $\varphi$, $\varepsilon_\rho$ approaches zero as $\rho$ does.
      Because the left-hand quantity in the display does not depend on $\rho$, by taking small $\rho$ we see that $\int_\R \phi_n\varphi$ must be arbitrarily small.
      Thus
      \begin{align*}
%        \lim_n \int_{x\in\R} \phi_n(x)\varphi(x)\,dx =
        \lim_n\langle\phi_n,\varphi\rangle=0=\varphi(0) \text{.}
      \end{align*}
    \end{proof}

    Having seen that any approximate identity approximates $\delta$ we hope to construct such a sequence.
    See that non-negative real Schwartz functions exist because the function
    \begin{equation}
      \label{eq:extestfcn}
      h(x) =
      \begin{cases}
        e^\frac{1}{x^2-1} &\text{if }|x|<1\\
        0 &\text{otherwise}
      \end{cases}
    \end{equation}
    is verifiable smooth at any $x\in\R$ and is identically zero off a bounded set, and so is Schwartz.
    Thus we may let $\phi$ be a non-negative Schwartz function having integral 1 and vanishing off of $[-A,A]$, so that for any $n\in\Z_{>0}$, the function $x\mapsto n\phi(nx) \call \phi_n$ is is a non-negative Schwartz function having integral 1 and vanishing off of $[-A/n,A/n]$.
%    Given any $\delta,\varepsilon>0$ consider the quantity
%    \begin{align*}
%      \int_{|x|\ge\delta} \phi_n(x) \,dx
%      &= \int_{|x|\ge n\delta} \phi(x) \,dx \text{.}
%    \end{align*}
%    Because $\phi$ is Schwartz, there is some $X$ so that $\int_{|x|\ge X}\phi(x)\,dx<\varepsilon$.
%    Thus for $n>X/\delta$ we have 
%    \begin{align*}
%      \int_{|x|\ge\delta} \phi_n(x) \,dx
%      &= \int_{|x|\ge n\delta} \phi(x) \,dx
%      \le \int_{|x|\ge X} \phi(x) \,dx
%      <\varepsilon \text{,}
%    \end{align*}
%    so for any $\delta,\varepsilon>0$ there is some $N$ so that $n>N$ implies $\int_{|x|\ge\delta}\phi_n(x)\,dx<\varepsilon$.
    Thus $\{\phi_n\}$ is an approximate identity.

    The fact that approximate identities exist implies a much stronger result---that \emph{any} tempered distribution may be approximated by Schwartz functions in the \ws.
    To see this, we must first recall a method for ``smoothing'' functions.

    Recall from \cref{defn:convolution} that the convolution of two Schwartz functions $f$ and $g$ is a function given by
    \begin{align*}
      (f*g)(x) = \int_{y\in\R} f(x-y)g(y)\,dy \text{.}
    \end{align*}
    Because $f$ and $g$ are Schwartz, see that $f*g$ is a smooth function of $x$ because the $x$ derivative passes through the $y$ integral and attaches itself to the $f$ term.
    Furthermore, as $|x|$ grows, we have that for any $y\in\R$ at least one of $|x-y|$ or $|y|$ is large, and so either $f(x-y)$ or $g(y)$ will approach zero.
    Thus we see that $(f*g)(x)$ will vanish as $|x|\rightarrow\infty$.
    In fact, multiplying the convolution by some power of $x$ we get
    \begin{align*}
      x^\ell(f*g)(x)
      &= \int_{y\in\R} x^\ell f(x-y)g(y)\,dy \text{.}
    \end{align*}
    For any fixed $y$ the function $x^\ell f(x-y)$ will vanish at infinity because it is Schwartz, and so the integral vanishes at infinity, because as $x\rightarrow\infty$ the integrand converges pointwise to the zero function.
    Thus the convolution of two Schwartz functions is another Schwartz function, and so \S acts on itself by convolution.

    It is almost immediate that this action allows approximation of any Schwartz function by convolution with an approximate identity:
    \begin{lemma}
      \label{lemma:approxbyconv}
      For an approximate identity $\{\phi_n\}$ and any $g\in\S$, the sequence of convolutions $\{\phi_n*g\}$ approximates $g$ in \S.
    \end{lemma}
    \begin{proof}
      Use \cref{thm:approxid} to calculate
      \begin{align*}
        (\phi_n*g)(x)
        &= \int_{y\in\R} \phi_n(y)g(x-y)\,dy
        \underset{n}{\rightarrow} g(x) \text{.}
      \end{align*}
    \end{proof}

    We hope to use a similar process to approximate tempered distributions.
    Happily, the action of \S on itself gives an action of \S on its embedded image in \SS.
    For some $f,g\in\S$, $f$ acts on the distribution $u_g$ arising from $g$ by
    \begin{align*}
      f*u_g = u_{f*g} \text{.}
    \end{align*}
    For any $\varphi\in\S$ we have
    \begin{align*}
      \langle f*u_g, \varphi\rangle
%      &= \langle u_{f*g}, \varphi\rangle
%      \\&= \int_{x\in\R} (f*g)(x)\varphi(x)\,dx
      &= \int_{x\in\R} \int_{y\in\R} f(x-y)g(y)\,dy\,\varphi(x)\,dx
      \\&= \int_{y\in\R} g(y) \int_{x\in\R} f(x-y)\varphi(x)\,dx\,dy
      \\&= \langle u_g, \tilde f*\varphi\rangle
    \end{align*}
    where $\tilde f(x) = f(-x)$.
    Again, this formula allows us to extend our definition of convolution to allow one of the arguments to be a distribution by an adjoint definition.
    Define the action of a Schwartz function $\phi$ on a distribution $u$ by the identity
    \begin{align*}
      \langle \phi*u, \varphi\rangle
      = \langle u, \tilde\phi*\varphi\rangle \text{.}
    \end{align*}
    Notably, unlike the convolution of Schwartz functions, this action is asymmetric, as we may not convolve two distributions.
    Thus we refer to the distribution $\phi*u$ as the \emph{mollification} of $u$ by $\phi$.

    We might wonder if a mollification is in fact given by integration against some classical function.
    Again letting $f,g\in\S$, write
    \begin{align*}
      (f*g)(x)
      &= \int_{y\in\R} f(x-y)g(y)\,dy
%      \\&= \int_{y\in\R} T_{-x}\tilde f(y) g(y)\, dy
      = \langle g, T_{-x}\tilde f\rangle
    \end{align*}
    which suggests a sensible expression for pointwise values of the mollification even when $g\in\SS$.
    For $u\in\SS$ and $\phi\in\S$ we wonder if the function
%    $\psi:\R\rightarrow \C$ given by $\psi(x)=\langle u, T_{-x}\tilde \phi\rangle$ gives rise to the distribution $\phi*u$ defined above.
    \begin{align*}
      \psi(x) = \langle u, T_{-x}\tilde\phi\rangle
    \end{align*}
    gives rise to the distribution $\phi*u$ defined above.
    Calculate for any $\varphi\in\S$
    \todo{fix this calculation}
    \begin{align*}
      \left\langle \psi, \varphi \right\rangle
      &= \int_{x\in\R} \langle u,T_{-x}\tilde\phi\rangle\varphi(x)\,dx
      = \left\langle u, \int_{x\in\R} (T_{-x}\tilde\phi)\varphi(x)\,dx \right\rangle
%      \\&= \left\langle u, \int_{x\in\R} \tilde\phi(y-x)\varphi(x)\,dx \right\rangle
      = \langle u, \tilde\phi*\varphi\rangle
    \end{align*}
    with the second equality exchanging the order of integration.
    Thus $\phi*u$ is in fact a classical function with pointwise values given by $(\phi*u)(x)=\psi(x)=\langle u, T_{-x}\tilde\phi\rangle$.
    Because $f\mapsto\tilde f$ is continuous (null sequences map to null sequences), translation is continuous (\cref{claim:translcont}), and $u$ is continuous, see that the composition
    \begin{align*}
      x, \phi \mapsto x, \tilde\phi \mapsto T_{-x}\phi \mapsto \langle u, T_{-x}\phi \rangle
    \end{align*}
    is a continuous function of $x\in\R$.
    Thus we may calculate that the derivative of $\phi*u$ at any $x\in\R$ is given by
    \begin{align*}
      \lim_{t\rightarrow0} \frac{\psi(x+t) - \psi(x)}{t}
      &= \lim_{t\rightarrow0} \frac{\langle u, T_{-x-t}\phi-T_{-x}\phi\rangle}{t}
      = \left\langle u, \lim_{t\rightarrow0} \frac{T_{-x-t}\phi-T_{-x}\phi}{t}\right\rangle
      = \left\langle u, T_{-x}\phi'\right\rangle
    \end{align*}
    and so $\phi*u$ is a smooth function because we may put all derivatives on the Schwartz function~$\phi$.
    Thus mollification is a sort of ``smoothing'' process on distributions---regardless of how strange the distribution $u$ is, the mollification $\phi*u$ is smooth enough to have pointwise values.
    Furthermore, this smoothing acts as we expect it to, in that we may approximate any distribution by convolution with approximate identity. 

    \begin{lemma}
      \label{lemma:mollwithid}
      For an approximate identity $\{\phi_n\}$ and some tempered distribution $u$, the mollification $\phi_n*u$ approximates $u$ in the \ws.
    \end{lemma}
    \begin{proof}
      For all $f\in\S$, calculate and apply \cref{lemma:approxbyconv} to see
      \begin{align*}
        \langle \phi_n*u, f\rangle
        &= \langle u, \phi_n*f\rangle
%        \\&= \left\langle u, x\mapsto \int_{y\in\R} \phi_n(x-y)f(y)\,dy \right\rangle
%        \\&= \left\langle u, x\mapsto \int_{y\in\R} \phi_n(-y)f(x+y)\,dy \right\rangle
%        \\&= \left\langle u, x\mapsto \int_{y\in\R} \phi_n(y)T_xf(-y)\,dy \right\rangle
%        \\&= \left\langle u, x\mapsto \int_{y\in\R} \phi_n(y)T_xf(y)\,dy \right\rangle
        = \left\langle u, \langle f, T_{-x}\phi_n\rangle \right\rangle
        = \left\langle u, \langle \phi_n, T_{x}f\rangle \right\rangle
        \\&\underset{n}{\longrightarrow} \left\langle u, T_xf(0) \right\rangle
        = \langle u,f\rangle \text{.}
      \end{align*}
    \end{proof}

    Any tempered distribution may be approximated by a sequence of smooth functions (mollifications).
    Unfortunately, arbitrary mollifications need not be Schwartz functions; they might only be fairly good functions that never vanish as $x$ grows.
    Thus we wonder whether arbitrary smooth functions may be approximated by Schwartz functions.
    To this end, consider the space \D of smooth functions with compact support (i.e. they are uniformly zero for sufficiently large inputs).\footnotemark
    \footnotetext{
      This is not the first time we have seen test functions; approximate identities are in fact sequences in \D.
      As was the case proving \cref{thm:approxid}, test functions are useful because their product with any function still has compact support.
    }
    Such functions are clearly Schwartz, and in fact form a dense\footnotemark subspace of \S, as we will see.
    \footnotetext{Recall that a subset $A$ of a topological space $X$ is called \emph{dense in $X$} if for any $x\in X$, every neighborhood of $x$ contains an element of $A$.}
    \D is called the space of \emph{test functions}; its dual $\D^*$ is a space of distributions that is larger than \SS.
    The function $h(x)$ from~\cref{eq:extestfcn} is a prototypical test function.
    
    A \emph{mesa function} is non-negative real test function that has constant value on a subset of the line and decays smoothly to zero outside that set.
    See that such functions exist because the integral of
    \begin{align*}
      \int_{-\infty}^x h(t)\,dt
    \end{align*}
    has value 0 for $x$ to the left of -1 and grows smoothly on $[-1,1]$ to a positive constant, which it takes for all $x>1$.
    Glueing this function to a translated and mirrored version of itself gives a mesa function.
    Just as approximate identities allow approximation of tempered distributions by smooth functions, mesa functions are used to approximate smooth functions with test functions.

    \begin{figure}[t]
      \begin{center}
        \begin{sagesilent}
          from matplotlib import ticker
          h = lambda x: e^(1/(x^2-1))
          g = Piecewise([[(.5,1), 0],[(1,2), h(x-2)/h(0)], [(2,5), 1], [(5,6), h(x-5)/h(0)], [(6,6.5), 0]])
          p = plot(g, color="black", plot_points=500, ticks=[[1, 2, 5, 6],[]], tick_formatter= ticker.FixedFormatter(['$-A_i-1$','$-A_i$','$A_i$','$A_i+1$']), figsize=[6,1])
        \end{sagesilent}
        \sageplot[width=\figwidth]{p}
      \end{center}
      \caption{An example mesa function $\psi_i$.}
      \label{fig:mesafunc}
    \end{figure}

    \begin{lemma}
      \label{lemma:ddenseins}
      The space of test functions is dense in the space of Schwartz functions.
    \end{lemma}
    \begin{proof}
      Let $\{A_i\}$ be a sequence of positive real numbers that approaches infinity and for any $i$ let $\psi_i$ be a mesa function that is identically 1 on $[-A_i,A_i]$ and identically zero outside $[-A_i-1,A_i+1]$, as shown in \cref{fig:mesafunc}.
      Given some $\varphi\in\S$, for any $i$ the function $\psi_i\varphi$ is a test function because it is zero for $|x|>A_i+1$ and pointwise multiplication of functions does not affect smoothness.
      Thus it remains only to show that $\{\psi_i\varphi\}\rightarrow\varphi$ in \S; this is the case whenever the sequence converges in every \Ckl.
      Recall
      \begin{align*}
        |\psi_i\varphi - \varphi|_\Ckl
        &= \sum_{0\le j\le k} \sup_{x\in\R} |(1+x^2)^\ell (\psi_i\varphi-\varphi)^{(j)}(x)|
        \\&\le \sum_{0\le j\le k} \sup_{x\in\R} |(\psi_i\varphi-\varphi)^{(j)}(x)|
%        &\le |\psi_i\varphi - \varphi|_\Coo
%        \le \sup_{|x|>A_i} |\varphi(x)|
      \end{align*}
      See that any derivative $(\psi_i\varphi-\varphi)^{(j)}$ is a sum has one term that is Schwartz ($\varphi^{(j)}$), and the remaining terms are the product of a Schwartz function with a test function.
      As $i$ grows we hope that all these terms vanish because the supremum of a sum is less than or equal to the sum of the suprema.
      First group the Schwartz-only term with the matching term $\psi_i\cdot\varphi^{(j)}$ and see that 
      \begin{align*}
        \sup_{x\in\R} |\psi_i(x)\varphi^{(j)}(x)-\varphi^{(j)}(x)|
        &= \sup_{x\in\R} |(\psi_i(x)-1)|\cdot|\varphi^{(j)}(x)|
        = \sup_{|x|>A_i} |(\psi_i(x)-1)|\cdot|\varphi^{(j)}(x)|
      \end{align*}
      approches zero as $i$ grows because $\psi_i-1$ remains bounded while $\varphi^{(j)}$ approaches zero.
      
      The remaining terms have the form
      \begin{align*}
        \sup_{x\in\R} |\varphi^{(m)}(x)\psi_i^{(n)}(x)|
      \end{align*}
      where $m+n=j$ and $n>0$.
      See that $\psi_i^{(n)}$ is uniformly zero on $[-A_i,A_i]$ and outside $[-A_i-1,A_i+1]$.
      For any $i$, $\psi_i^{(n)}$ is bounded, and as $i$ grows $\sup\limits_{|x|\in[A_i,A_i+1]}\varphi^{(m)}(x)$ dies faster than any power of $A_i$.
      Thus the term goes to zero as long as the bound of $\psi_i^{(n)}$ is slowly increasing w.r.t. $A_i$ as $i$ grows, which is at least assured by careful choices of $\psi_i$. \todo{clear this up} 

      Thus \D is dense in \S.
    \end{proof}

    \begin{lemma}
      \label{lemma:ddenseinssfromsmooth}
      Let $u_f$ be a tempered distribution given by integrating against a smooth function $f$.
      Then there is a sequence of test functions $\{\phi_i\}$ that gives (via integration-against) a sequence of distributions approximating $u_f$ in the \ws.
    \end{lemma}
    \begin{proof}
      Let $\{\psi_i\}$ be a sequence of mesa functions as above, and define $\phi_i$ by $\phi_i(x) = \psi_i(x)f(x)$.
      See that each $\phi_i$ is test because it is smooth and vanishes identically for large $|x|$.
      Identifying the $\phi_i$ with the corresponding tempered distributions, calculate for any $\varphi\in\S$
      \begin{align*}
        \langle \phi_i, \varphi\rangle
        &= \int_{x\in\R} \psi_i(x)f(x)\varphi(x)\,dx
        = \langle u_f, \psi_i\varphi\rangle
        \underset{i}{\rightarrow} \langle u_f, \varphi\rangle
      \end{align*}
      by the continuity of evaluation and \cref{lemma:ddenseins}
    \end{proof}

    \begin{thm}
      \label{thm:sdenseinss}
      The space of Schwartz functions is dense in the space of tempered distributions.
    \end{thm}
    \begin{proof}
      Let $u\in\SS$.
      Now pick a neighborhood $U$ of $u$ in \SS.
      Because the sequence $\{\phi_n*u\}\rightarrow u$ by \cref{lemma:mollwithid} there is some $n$ so that $\phi_n*u\in U$.
%      Then because the sequence $\{\psi_i(\phi_n*u)\}\rightarrow\phi_n*u$ there is some $i$ so that $\psi_i(\phi_n*u)\in U$.
      By \cref{lemma:ddenseinssfromsmooth}, there is some mesa function $\psi$ so that $\psi(\phi_n*u)\in U$.
      Because $\psi(\phi_n*u)$ is test (and thus Schwartz) and $u$ and $U$ were arbitrary, \S is dense in \SS.
    \end{proof}

\end{document}
