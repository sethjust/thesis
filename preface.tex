\chapter*{Preface}
  The theory of generalised functions (or ``distributions'') is a development of the heuristic methods of the operational calculus developed in the late nineteenth century, which made use of divergent series and integrals to solve differential equations.
  These methods were frowned upon by the mathematical establishment, and were not formalized until the later work of Sobolev, Bochner, Schwartz and many others in the first half of the twentieth century.
  Thanks to their work these techniques are now accepted mathematics.

  This thesis attempts to present the subject in modern language.
  While the scope is limited, the presentation relies on as few specific details as possible with the hope of introducing the tools necessary to develop the theory of distributions on more exotic domains than \R.
  Throughout the text the reader is presumed to be comfortable with real analysis, including Lebesgue integration on \R, point-set topology, and the theory of topological vector spaces and their duals.
  Nonetheless, the exposition emphasizes characterization of objects by active properties, de-emphasizing the details of the machinery at play, and exposing the relevant issues.
