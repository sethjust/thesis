\chapter{Test Functions and Schwartz Functions}
%  [[Figure out what sort of topological space [test fcns](\R) is]]
%  [[Figure out what sort of topological space [Schwartz fcns](\R) is]]
  
  \section{Preliminaries}

    The set of continuous (complex-valued) functions defined on the real line, $C^0(\R)$, naturally forms a vectorspace over \R, with addition and scalar multiplication defined pointwise: 
    \begin{align*}
      (f+g)(x) &= f(x)+g(x)\\
      (a\cdot f)(x) &= a\cdot f(x)
    \end{align*}
%    % hilbert is normed vspace, banach is complete hilbert
%    We would like to imbue this space with a metric that allows us to (sensibly) discuss limits of functions, and thus make it a complete topological (Banach?) space.
%    %pntws limits of cnts are not always cnts
%    To this end we introduce the \emph{sup norm} and its associated metric:
%    \begin{align*}
%      |f|_{C^0} &= \sup_{x\in\R}|f(x)|\\
%      d(f,g) &= |f-g|_{C^0}
%    \end{align*}
%%    Unfortunately, applied to an arbitrary function, the sup norm may be infinite, limiting its utility. 
%%    To remedy this, we note that the Extreme Value Theorem ensures that the sup norm is finite for any function with \emph{compact support}, \emph{i.e.}, one whose value is zero outside of some compact subset of the line.
%    We define $C^0_c(\R)$ to be the space of %such
%    compactly supported continuous functions.
%%    The simplicity of the following proof illustrates the naturalness of the sup norm applied to $C^0_c(\R)$.
%%
%%    \begin{claim}
%%      For $x\in\R$, the evaluation functional $C^0_c(\R)\rightarrow\C$ given by $f\mapsto f(x)$ is continuous.
%%      \begin{proof}
%%        Given $\varepsilon > 0$, $d(f,g)<\varepsilon \Rightarrow |f(x)-g(x)|<\varepsilon$, showing continuity.
%%      \end{proof}
%%    \end{claim}
%
%    However, $C^0_c(\R)$ is not complete in the sup norm topology; for example, a sequence of wider-but-shorter spikes will be Cauchy, but the limit of such functions lacks compact support. %TODO: make more concrete
%    Completing $C^0_c(\R)$ with respect to the sup norm topology gives $C_0^0(\R)$, the space of continuous functions \emph{vanishing at infinity} (Explain why). %TODO
%
%    Similarly, the space of $k$-times (continuously) differentiable functions with compact support, $C_c^k(\R)$, is not complete in the topology given by
%    \begin{equation*}
%      |f| = \sup_{j\le k} \sup_{x\in\R} |f^{(j)}(x)|
%    \end{equation*}
%    as once again, support may leak out to infinity.
%    %TODO: motivate this norm
%    The completion of $C_c^k(\R)$ with regard to this topology is $C_0^k(\R)$, the space of functions whose first $k+1$ derivatives (including the $0^\text{th}$) are continuous and vanish at infinity (Again, explain why). %TODO

    \begin{defn}[smooth function]
      A function $f\in C^0(\R)$ is called \emph{smooth} when it has infinitely many continuous derivatives.
    \end{defn}
    \begin{defn}[support]
      The \emph{support} of a function $f\in C^0(\R)$, denoted $\spt f$, is the set $\{x\in\R: f(x)\ne0\}$.
      When $\spt f$ is a compact subset of \R, $f$ is said to have \emph{compact support}.
    \end{defn}
    \begin{defn}[test function]
      A function $f\in C^0(\R)$ is a \emph{test function} if it is smooth and has compact support.
      The set of all test functions is denoted \D.
    \end{defn}

    We would like to give \D the structure of a topological space, where we can make use of concepts like limits and the continuity of mappings.
