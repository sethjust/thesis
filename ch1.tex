\chapter{Test Functions}
%  [[Figure out what sort of topological space [test fcns](\R) is]]
%  [[Figure out what sort of topological space [Schwartz fcns](\R) is]]
  
  \section{Preliminaries}

    As explained in the Introduction, test functions are our window into the world of distributions.
    As such, our choice of what is or is not a test function will change our concept of what is or is not a distribution.
    However, we do not have a wholly free choice in our selection of test functions -- we need to ensure that every function we care to consider qualifies as a distribution.
    This assertion could fail if we allow a function $f$ and a test function $\varphi$ where the integral $\int_{-\infty}^{\infty} f(x)\varphi(x)dx$ is not defined.
    Thus allowing $f$ to be \emph{any} conceivable function is clearly impossible; we must restrict our view of what is a valid function. %TODO: add example
    As a starting point we add only one requirement to our functions: that they be \emph{piecewise continuous}. %TODO: be more precise
    As we will see, this restriction is appropriately severe: it allows us to have a sufficiently broad space of test functions to be effective while still ensuring it is small enough to be manageable.
    However, we still have some freedom in our definition of test functions.

    Particularly, we want to choose a space of test functions that has a nice structure, which will in turn allow us to more easily understand the space of distributions.
    The particular structure that we would like to use is that of a \emph{topological space}.
    Because a distribution is simply a mapping from test functions to (complex) numbers, we can then view the space of distributions as the \emph{dual space} of test functions, a well understood mathematical concept.

    Viewing the set of test functions as having structure is a strange concept in some ways, but it is made easier as we restrict our view.
    Let us start by assuming that test functions are continuous.
    The set of continuous (complex-valued) functions defined on the real line, $C^0(\R)$, naturally forms a vectorspace over \R, with addition and scalar multiplication defined pointwise: 
    \begin{align*}
      (f+g)(x) &= f(x)+g(x)\\
      (a\cdot f)(x) &= a\cdot f(x)
    \end{align*}
    To make $C^0(\R)$ a topological space we can imbue it with a concept of distance.
    %pntws limits of cnts are not always cnts
    % hilbert is normed vspace, banach is complete hilbert
    To this end we introduce the \emph{sup norm} and its associated metric:
    \begin{align*}
      |f|_{C^0} &= \sup_{x\in\R}|f(x)|\\
      d(f,g) &= |f-g|_{C^0}
    \end{align*}
    The simplicity of the following proof illustrates the naturalness of the sup norm applied to $C^0(\R)$.

    \begin{claim}
      For $x\in\R$, the evaluation functional $C^0(\R)\rightarrow\C$ given by $f\mapsto f(x)$ is continuous.
      \begin{proof}
        Given $\varepsilon > 0$, $d(f,g)<\varepsilon \Rightarrow |f(x)-g(x)|<\varepsilon$, showing continuity.
%        Given $\varepsilon > 0$, $d(f,g)<\varepsilon \Rightarrow |f(x)-g(x)|<\varepsilon$ because $\forall x$ $|f(x)-g(x)|\le\sup_{y\in\R}|f(y)-g(y)|=d(f,g)$.
%        Thus the evaluation functional is continuous.
      \end{proof}
    \end{claim}

    Unfortunately, even with the sup norm, $C^0(\R)$ fails to be a topological space, as the norm is not defined for all continuous functions, as it might be infinite. 
    To remedy this we must once again restrict the space we are considering.
    \begin{defn}
      The \emph{support} of a function $f:\R\rightarrow\C$, denoted $\spt f$, is the set $\{x\in\R: f(x)\ne0\}$.
      When $\spt f$ is a compact subset of \R, $f$ is said to have \emph{compact support}.
    \end{defn}
    
    Note that the Extreme Value Theorem ensures that the sup norm is finite for any function with compact support, \emph{i.e.}, one whose value is zero outside of some compact subset of the line.
    Define $C^0_c(\R)$ to be the space of such compactly supported continuous functions.

    However, $C^0_c(\R)$ is not complete in the sup norm topology; for example, a sequence of wider-but-shorter spikes will be Cauchy, but the limit of such functions lacks compact support. %TODO: make more concrete
    Completing $C^0_c(\R)$ with respect to the sup norm topology gives $C_0^0(\R)$, the space of continuous functions \emph{vanishing at infinity}. (Explain why!) %TODO

    %TODO: transition
    Similarly, the space of $k$-times (continuously) differentiable functions with compact support, $C_c^k(\R)$, is not complete in the topology given by
    \begin{equation*}
      |f| = \sup_{j\le k} \sup_{x\in\R} |f^{(j)}(x)|
    \end{equation*}
    as once again, support may leak out to infinity.
    %TODO: motivate this norm; comment on equivalence to \sup\sum norm
    The completion of $C_c^k(\R)$ with regard to this topology is $C_0^k(\R)$, the space of functions whose first $k+1$ derivatives (including the $0^\text{th}$) are continuous and vanish at infinity. (Again, explain why!) %TODO

%    \begin{defn}[smooth function]
%      A function $f\in C^0(\R)$ is called \emph{smooth} when it has infinitely many continuous derivatives.
%    \end{defn}
%    \begin{defn}[test function]
%      A function $f\in C^0(\R)$ is a \emph{test function} if it is smooth and has compact support.
%      The set of all test functions is denoted \D.
%    \end{defn}
%    We would like to give \D the structure of a topological space, where we can make use of concepts like limits and the continuity of mappings.
