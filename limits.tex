\documentclass[thesis.tex]{subfiles}
\begin{document}
\ifdraft{
  \fancypagestyle{plain}{%
    \fancyhead{}
    \renewcommand\headrulewidth{0pt}
    \fancyfoot[CO,CE]\draftid
  }
  \fancyfoot[CO,CE]\draftid
}{ }
\onehalfspacing
% Change math display spacing here, so spacing commands don't reset it.
\abovedisplayskip=3pt
\belowdisplayskip=3pt
\abovedisplayshortskip=2pt
\belowdisplayshortskip=2pt

  \appendix
  \chapter{Products and Limits of Topological Spaces}
    \label{ch:topprod}

    In \cref{sec:spcoffunc} we constructed a space of functions with finitely many derivatives and finitely fast decay and showed that this space can be given the topological structure of a Banach space.
    We hope that this work will allow us to find useful topological structure for the space of Schwartz functions, which have \emph{infinitely} many derivatives and infinitely fast decay.
    This is exactly the process of taking a limit---we want to say $\S=\lim\limits_{k,\ell}C^k_\ell$, but this tells us nothing about the \emph{topology} of such a limit.
    Rather than laboriously \emph{construct} such a topology, we hope that by \emph{characterizing} the behavior of the limit object we can see what the topology on \S must be.

    This appendix develops general facts about limits of topological spaces.
    First we will define the \emph{product} of topological spaces; we will see that the \emph{limit} of such spaces will be a closed subspace of the corresponding product.

    \section{Products of Topological Spaces}
    We define the product of a collection of topological spaces by a mapping property; this definition will \emph{force} the product to assume a unique topology.
      \begin{defn}
        Consider a collection of non-empty topological sets $X_\alpha$, indexed by $\alpha\in A$, some (possibly uncountable) index set.
        We call a topological space $X$ a \emph{product} of the $X_\alpha$ if there are continuous \emph{projection maps} 
        \begin{align*}
          p_\alpha:X\longrightarrow X_\alpha
        \end{align*}
        such that for all topological spaces $Z$ with continuous maps $f_\alpha:Z\rightarrow X_\alpha$ there is a unique continuous map 
        \begin{align*}
          f:Z\longrightarrow X
        \end{align*}
        satisfying $f_\alpha=p_\alpha\circ f$ for all $\alpha$.
        (We say that $f_\alpha$ \emph{factors through} the $p_\alpha$ uniquely.)
      \end{defn}
      Pictorially, the definition of a product says that all triangles commute in the diagram
      \begin{displaymath}
        \xymatrix{
          && X \ar@/_/[dll] \ar@/_/[dl]^{p_\alpha} \ar@/^/[dr]_{p_\beta} \ar@/^/[drr] \\
          \cdots & X_\alpha && X_\beta & \cdots\\
          && Z \ar@/^/[ull] \ar@/^/[ul]_{f_\alpha} \ar@{-->}[uu]_f \ar@/_/[ur]^{f_\beta} \ar@/_/[urr]
         }
      \end{displaymath}

      \begin{claim}
        \label{claim:uniqprod}
        The product is \emph{unique}, up to (continuous) isomorphism.
        \begin{proof}
          Suppose we have a collection of topological spaces $X_\alpha$ and two products, $X$ with projection maps $p_\alpha$ and $Y$ with projection maps $q_\alpha$.
          Considering $X$ and putting $Y$ in the position of $Z$ from the definition we have a unique $q$ so that the triangles commute in
          \begin{displaymath}
            \xymatrix{
              && X \ar@/_/[dll] \ar@/_/[dl]^{p_\alpha} \ar@/^/[dr]_{p_\beta} \ar@/^/[drr] \\
              \cdots & X_\alpha && X_\beta & \cdots\\
              && Y \ar@/^/[ull] \ar@/^/[ul]_{q_\alpha} \ar@{-->}[uu]_q \ar@/_/[ur]^{q_\beta} \ar@/_/[urr]
             }
          \end{displaymath}
          so $q_\alpha=p_\alpha\circ q$.
          Reversing the positions of $X$ and $Y$ we have a unique $p:X\rightarrow Y$ so that $p_\alpha=q_\alpha\circ p$.
          Thus the map $p\circ q:X\rightarrow X$ makes the triangles commute in
          \begin{displaymath}
            \xymatrix{
              && X \ar@/_/[dll] \ar@/_/[dl]^{p_\alpha} \ar@/^/[dr]_{p_\beta} \ar@/^/[drr] \\
              \cdots & X_\alpha && X_\beta & \cdots\\
              && X \ar@/^/[ull] \ar@/^/[ul]_{p_\alpha} \ar@{-->}[uu]_{p\circ q} \ar@/_/[ur]^{p_\beta} \ar@/_/[urr]
             }
          \end{displaymath}
          Because $X$ is a product of the $X_\alpha$ by hypothesis, the map $p\circ q$ is unique.
          Trivially the identity mapping satisfies the requirements for $p\circ q$ (and symmetrically for $q\circ p$).
          By uniqueness $p\circ q=\id$ and $q\circ p=\id$,  so $X$ and $Y$ are (continuously) isomorphic.
        \end{proof}
      \end{claim}

      Now that we have shown that our mapping-property definition gives a unique product (provided one exists), we hope to investigate properties of this product.
      From our definition, we hope that by considering various sets $Z$ and maps $f_\alpha$ we can learn about $X$ (supposing that it exists).

      First, the product has no fewer elements than we expect; if we pick an element from each factor, we may find an element of the product that projects to each of those elements.
      \begin{claim}
        For any set $\{x_\alpha:\alpha\in A\}$ with $x_\alpha\in X_\alpha$ there exists some $x\in X$ such that $p_\alpha(x)=x_\alpha$ for all $\alpha$.
        \begin{proof}
          Take the one-element set $Z=\{z\}$ with maps $f_\alpha(z)=x_\alpha$.
          Then 
          \begin{align*}
            p_\alpha(f(z))=f_\alpha(z)=x_\alpha \text{.}
          \end{align*}
          The element $x=f(z)$ is the desired one.
        \end{proof}
      \end{claim}
      
      The product also has no more elements than we could reasonably expect; that is, we see that distinct points in the product project into some factor distinctly. 
      \begin{claim}
%        For $x,y\in X$, $x\ne y$, there is some $\alpha\in A$ such that $p_\alpha(x)\ne p_\alpha(y)$.
        Let $x,y\in X$ have $p_\alpha(x)\ne p_\alpha(y)$.
        Then $x\ne y$.
        \begin{proof}
%          First, suppose that $p_\alpha(x)=p_\alpha(y)$ for all $\alpha$.
%          Now let $Z$ be the one-element set $\{z\}$, and define $f_\alpha(z)=p_\alpha(x)$.
%          By hypothesis, there exists some map $f:Z\rightarrow X$ satisfying $f_\alpha = p_\alpha \circ f$.
%          Because maps from one-element sets are wholly determined by their images, we see that $p_\alpha(f(z))=f_\alpha(z)$, so $p_\alpha(f(z))=p_\alpha(x)$.
%          But $f(z)=x$ \emph{and} $f(z)=y$ satisfy this condition, contradicting the uniqueness of the mapping $f$.
          Again, let $Z=\{z\}$ and define $f_\alpha(z)=p_\alpha(x)$.
          Then $f(z)=x$ gives the unique induced map, and by definition $f(z)$ cannot equal $y$ (as by hypothesis $f$ could not factor through every $p_\alpha$ if it were the case), proving the claim.
        \end{proof}
      \end{claim}

      These results suggest that as a set (that is, without consider topology) we may construct the product as we might expect.
      \begin{claim}
        \label{claim:prodiscart}
        As a \emph{set} the product of topological spaces is the normal (Cartesian) product
        \begin{align*}
          X = \{\{x_\beta:\beta\in A\}: x_\beta\in X_\beta\}
        \end{align*}
        with projections
        \begin{align*}
          p_\alpha(\{x_\beta:\beta\in A\}) = x_\alpha\text{.}
        \end{align*}
        \begin{rmk}
          For this proof we make no reference to topology (continuity of maps).
          By working in the category of \emph{sets} we prove that the product is the Cartesian one \emph{regardless of topology}.
          This set-theoretic construction proves that we may construct limits as sets, leaving us to prove only the existence of a suitable topology later.
        \end{rmk}
        \begin{proof}[Proof of \cref{claim:prodiscart}]
          Consider a collection of $X_\alpha$, and suppose they have a product $X$ as above.
          For a set $Z$ and some collection of maps $f_\alpha:Z\rightarrow X_\alpha$, the map $f:Z\rightarrow X$ given by
          \begin{align*}
            f(z) = \{f_\alpha(z):\alpha\in A\}
          \end{align*}
          will be compatible with the $p_\alpha$, and is visibly the only such map.
        \end{proof}
      \end{claim}

    \section{Product Topology}
      We now turn our attention to the \emph{topology} on a product of topological spaces.
      We hope that we can construct the appropriate topology on a product of spaces using only the topologies on those spaces and our mapping property definition.
      Recall that a map $f:A\rightarrow B$ is continuous iff the inverse image of open is open; that is, for any open $u\subseteq B$, the set
      \begin{align*}
        f^{-1}(u) = \{x\in A:f(x)\in u\}
      \end{align*}
      is open in $A$.

      Recall that our definition of the product requires that all the maps be continuous.
      So, for any $u_\alpha$ open in $X_\alpha$, $p_\alpha^{-1}(u_\alpha)$ is open in $X$.
      Thus the topology on $X$ contains the topology generated by $\{p_\alpha^{-1}(u_\alpha):\alpha\in A\text{ and } u_\alpha\text{ open in }X_\alpha\}$.
      We say that the topology on $X$ is ``no coarser than'' this topology, as it cannot contain fewer sets.
      
      Now, given a family of continuous functions $f_\alpha:Z\rightarrow X_\alpha$, the (unique) induced map $f:Z\rightarrow X$ must be continuous, so for all $u$ open in $X$, $f^{-1}(u)$ is open in $Z$.
      Thus the topology on $X$ must be sufficiently coarse to satisfy $f^{-1}(u)$ being open in $Z$ for all $Z$ and $f$.
      Ideally, the topology given by the $X_\alpha$ is coarse enough to satisfy continuity of $f$.

      \begin{thm}
        The topology generated by 
        \begin{align*}
          \{p_\alpha^{-1}(u_\alpha):\alpha\in A\text{ and } u_\alpha\text{ open in }X_\alpha\}
        \end{align*}
        is coarse enough to be the product topology on $X$.
      \end{thm}
      \begin{proof}
        By the properties of inverse images it is sufficient to show that the inverse image under $f$ of each element of the generating set is open in $Z$.
        Recall that $p_\alpha\circ f=f_\alpha$, so 
        \begin{align*}
          f^{-1}(p_\alpha^{-1}(u_\alpha)) = f_\alpha^{-1}(u_\alpha)
        \end{align*}
        See that each of the $f_\alpha$ are continuous, so for $u_\alpha$ open in $X_\alpha$, the set $f_\alpha^{-1}(u_\alpha)$ is open in $Z$.
        Thus each $f^{-1}(p_\alpha^{-1}(u_\alpha))$ is open in $Z$.
      \end{proof}

      \begin{rmk}
        The key point here was the \emph{existence} of a topology on the product that is consistent with our definition. 
        The topology generated from the $X_\alpha$ might have been \emph{too fine} for the map $f$ to be continuous, but we this is not the case.
        Having shown existence, the continuity of the isomorphism in \cref{claim:uniqprod} then shows that \emph{any} product with \emph{any}  topology (provided that it is consistent with the product definition) will be homeomorphic to the Cartesian product with the topology given above.

        Notably (at least when we have infinitely many factors), the product topology is \emph{not} the box topology, which is generated by Cartesian products of an open set in each factor space.
        This is because an open set in the box topology is an infinite intersection of opens in the product topology.
        Thus, in general, the box topology is finer than the product topology.
      \end{rmk}

    \section{Projective Limits}
      We now turn our attention to limits of topological spaces.
      Once again, we characterize the limit by a mapping property before constructing it.
      \begin{defn}[Projective Limit]
        \label{defn:projlim}
        Consider a collection of topological spaces $X_i$ with continuous \emph{transition maps} $\varphi_{i,i-1}:X_i\rightarrow X_{i-1}$:
        \begin{displaymath}
          \xymatrix{
            \cdots \ar[r]^{\varphi_{n+1,n}} & X_n \ar[r]^{\varphi_{n,n-1}} & X_{n-1} \ar[r]^{\varphi_{n-1,n-2}} & \cdots \ar[r]^{\varphi_{21}} & X_1 \ar[r]^{\varphi_{10}} & X_0
          }
        \end{displaymath}
        The \emph{projective limit} of the $X_i$, denoted $\lim\limits_i X_i$ (suppressing reference to the transition maps) is a topological space $X$ and maps $\varphi_n:X\rightarrow X_n$ that are \emph{compatible} with the transition maps in the sense that 
        \begin{align*}
          \varphi_{n-1}=\varphi_{n,n-1}\circ\varphi_n
        \end{align*}
        so the triangles commute in
        \begin{displaymath}
          \xymatrix{
            X \ar@/^20pt/[rr]_{\varphi_1} \ar@/^30pt/[rrr]^{\varphi_0}
            & \cdots \ar[r]^{\varphi_{21}}
            & X_1 \ar[r]^{\varphi_{10}}
            & X_0
          }
        \end{displaymath}
        We require that for \emph{any} space $Z$ with maps $f_n:Z\rightarrow X_n$ that are compatible with the maps~$\varphi_{n,n-1}$ there is a unique $f:Z\rightarrow X$ through which the $f_n$ factor.
        Thus, whenever the triangles commute in
        \begin{displaymath}
          \xymatrix{
             \cdots \ar[r]^{\varphi_{21}}
            & X_1 \ar[r]^{\varphi_{10}}
            & X_0
            \\ \cdots \\
            Z \ar[uur]^{f_1} \ar[uurr]^{f_0}
          }
        \end{displaymath}
        there is a unique map $f$ that makes the triangles commute in
        \begin{displaymath}
          \xymatrix{
            X \ar@/^20pt/[rr]_{\varphi_1} \ar@/^30pt/[rrr]^{\varphi_0}
            & \cdots \ar[r]^{\varphi_{21}}
            & X_1 \ar[r]^{\varphi_{10}}
            & X_0
            \\ & \cdots \\
            & Z \ar@{-->}[uul]^f \ar[uur]_{f_1} \ar[uurr]_{f_0}
          }
        \end{displaymath}
      \end{defn}
      \begin{claim}
        The projective limit is unique up to (continuous) isomorphism.
        \begin{proof}
          As in the proof of the uniqueness of the product, we see that for any two limits $X$ and $Y$ of the same objects there are unique $p:X\rightarrow Y$ and $q:Y\rightarrow X$.
          By the uniqueness of the induced map in our definition, we then see that $p\circ q=\id$ and so $X$ and $Y$ are homeomorphic.
        \end{proof}
      \end{claim}

      Having defined the (projective) limit of topological spaces in terms of a mapping-property, we hope that such an object does exist.
      \begin{thm}
        \label{claim:limitexists}
        Consider a countably-indexed collection of topological spaces $X_i$ with continuous transition maps $\varphi_{i,i-1}$.
        Suppose $Y$, with projection maps $p_i$, is the product of the $X_i$.
        Then the limit $X$ of the $X_i$ exists as a subspace of $Y$.
        Namely, $X$ is where the transition maps and projections are compatible, so
        \begin{align*}
          X = \{x\in Y: \varphi_{i,i-1}(p_i(x)) = p_{i-1}(x) \text{ for all } i\}
        \end{align*}
        with the maps $\varphi_i:X\rightarrow X_i$ given by restrictions of the projections from the product:
        \begin{align*}
          \varphi_i = p_i|_X :X\rightarrow X_i
        \end{align*}
      \end{thm}
      \begin{proof}
        Let $j$ be the inclusion map $X\rightarrow Y$.
        Thus we have the diagram
        \begin{displaymath}
          \xymatrix{
            X \ar[r]^j \ar@/^37pt/[rrr]_{\varphi_1} \ar@/^50pt/[rrrr]^{\varphi_0}
            & Y \ar@/^20pt/[rr]_{p_1} \ar@/^30pt/[rrr]^{p_0}
            & \cdots
            & X_1
            & X_0
          }
        \end{displaymath}
        of commuting (curvy) triangles.
        However, the maps $p_i$ do not respect the transition maps of our limit, and so we have the diagram
        \begin{displaymath}
          \xymatrix{
            X \ar[r]^j \ar@/^37pt/[rrr]_{\varphi_1} \ar@/^50pt/[rrrr]^{\varphi_0}
            & Y \ar@/^20pt/@{..>}[rr]_{p_1} \ar@/^30pt/@{..>}[rrr]^{p_0}
            & \cdots \ar[r]^{\varphi_{21}}
            & X_1 \ar[r]^{\varphi_{10}}
            & X_0
          }
        \end{displaymath}
        where solid triangles commute, but not necessarily triangles with dotted edges.
        
        Now, for any space $Z$ and \emph{any} family of maps $f_i$, we have, from the definition of the product, a unique map $F$ so that all triangles commute in the diagram
        \begin{displaymath}
          \xymatrix{
            X \ar[r]^j \ar@/^37pt/[rrr]_{\varphi_1} \ar@/^50pt/[rrrr]^{\varphi_0}
            & Y \ar@/^20pt/[rr]_{p_1} \ar@/^30pt/[rrr]^{p_0}
            & \cdots
            & X_1
            & X_0
            \\ && \cdots \\
            && Z \ar@{-->}[uul]^F \ar[uur]_{f_1} \ar[uurr]_{f_0}
          }
        \end{displaymath}
        However, because we are concerning ourselves with the limit, we suppose that the $f_i$ are compatible with the transition maps $\varphi_{i,i-1}$, so that $f_{i-1} = \varphi_{i,i-1}\circ f_i$.
        We hope that we can find a unique $f:Z\rightarrow X$.
        Taking $z\in Z$, we see that
        \begin{align*}
          p_{i-1}(F(z)) = f_{i-1}(z) = \varphi_{i,i-1}(f_i(z)) = \varphi_{i,i-1}(p_i(F(z)))
        \end{align*}
        By our definition of $X$, $F(z)\in X$, and so $F(Z)\subseteq X$.
        Thus define $f:Z\rightarrow X$ by $f(z) = F(z)$, so we have the diagram
        \begin{displaymath}
          \xymatrix{
            X \ar@{-->}[r]^j \ar@{-->}@/^37pt/[rrr]_{\varphi_1} \ar@{-->}@/^50pt/[rrrr]^{\varphi_0}
            & Y \ar@/^20pt/@{..>}[rr]_{p_1} \ar@/^30pt/@{..>}[rrr]^{p_0}
            & \cdots \ar@{~>}[r]^{\varphi_{21}}
            & X_1 \ar@{~>}[r]^{\varphi_{10}}
            & X_0
            \\ && \cdots \\
            && Z \ar[uull]^f \ar[uul]^F \ar[uur]_{f_1} \ar[uurr]_{f_0}
          }
        \end{displaymath}
        where
        \begin{multicols}{2}
          \begin{itemize}
            \item dashed-and-dotted triangles commute;
            \item solid-and-dotted triangles commute;
            \item solid-and-dashed triangles commute;
            \item wavy-and-dashed triangles commute;
            \item wavy-and-solid triangles commute.
          \end{itemize}
        \end{multicols}
        Because the map $F$ is unique by hypothesis, we see that $f$ is unique, and so the compatibility conditions on the $f_i$ cause them to factor through the subspace $X$ of the product.
      \end{proof}

      Our use of mapping-property characterizations to define the product and limit allows us to now infer the topology of the limit from that of the product.
      Particularly, the (implied) continuity of the maps in our definition allows us to see what the topology must be.
      The result is the \emph{subspace topology}: for a subspace $X$ of a topological space $Y$, a set~$U\subseteq X$ is open in $X$ iff there is some $V$ open in $Y$ such that $U=X\cap V$.%
      \footnote{We can define the subspace topology by a mapping-property characterization: we require that inclusion $i:X\rightarrow Y$ is continuous, and for any space $Z$ and continuous $F:Z\rightarrow Y$ satisfying $F(Z)\subset X$, there is a unique $f:Z\rightarrow X$ such that $f=i\circ F$.}

      \begin{claim}
        Let $X$ (with transition maps $\varphi_i$) be the limit of topological spaces $X_i$.
        The topology on $X$ is generated by
        \begin{align*}
          \{\varphi_i^{-1}(u_i):u_i\text{ is open in } X_i\}\text{.}
        \end{align*}
        \begin{proof}
          Let $U$ be an open subset of $X$.
          Letting $Y=\prod_i X_i$, there is some open subset $V$ of $Y$ such that $U=X\cap V$.
          Now recall that for $V$ to be open in the product topology means it is open in the topology generated by $\{p_i^{-1}(u_i):u_i\text{ open in }X_i\}$, where $p_i$ is the~$i^\text{th}$ projection $Y\rightarrow X_i$.
          Thus $V$ is an arbitrary union of finite intersections of sets of the form $p_i^{-1}(u_i)$.
          Because the inverse image of a union is the union of inverse images (i.e. $f^{-1}(A\cup B)=f^{-1}(A)\cup f^{-1}(B)$), and likewise with intersections, we may suppose WLOG that $V=p_i^{-1}(u_i)$ for some $i$ and $u_i$ open in $X_i$.
      
          Denote the inclusion $X\rightarrow Y$ by $j$.
          Because $\varphi_i=p_i|_X$, see that $p_i\circ j = \varphi_i$, so for any~$v\subseteq X_i$, $\varphi_i^{-1}(v) = p_i^{-1}(v)\cap X$.
          Thus $U=\varphi_i^{-1}(u_i)$.
          In full generality, we see that any open set in $X$ is generated by (arbitrary unions and finite intersections of) sets of this form, proving the claim.
        \end{proof}
      \end{claim}

    \section{Indexing by Larger Sets}
      Our present definition of limits allows indexing only by a countable set.
      However, we need to allow more general schemes of ordering limitands in order to take interesting limits.
      Particularly, in taking the limit of the Banach Spaces $C^k_\ell$ with regard to $k$ \emph{and} $\ell$ we hope that a doubly-indexed limit shares the pleasant behaviour of the singly-indexed variety.
      
      Recall that a \emph{partial ordering} on a set $A$ is a relation $>$ that is antisymmetric (so $a>b$ implies $b\not>a$)\footnote{and thus antireflexive: $a\not>a$.} and transitive (so $a>b$ and $b>c$ implies $a>c$).
      A set with a partial ordering is called a \emph{poset}.

      A \emph{projective system} indexed by a poset $I$ is a collection of objects $X_i$ for $i\in I$, along with transition maps $\varphi_{ij}:X_i\rightarrow X_j$ for all $i,j\in I$ with $i>j$.
      These maps must be \emph{compatible}, so for all $i>j>k$ in $I$, we have the commutative triangle
      \begin{displaymath}
        \xymatrix{
          X_i \ar[r]^{\varphi_{ij}} \ar[dr]_{\varphi_{ik}} & X_j \ar[d]^{\varphi_{jk}} \\
          & X_k
        }
      \end{displaymath}

      The limit of a projective system is a set $X$ with projection maps $p_i$ for all $i\in I$ that are compatible with the transition maps, so for $i>j$ we have the commutative triangle
      \begin{displaymath}
        \xymatrix{
          X \ar[rr]^{p_i} \ar[drr]_{p_j} && X_i \ar[d]^{\varphi_{ij}} \\
          && X_j
        }
      \end{displaymath}
      \emph{And} for any space $Z$ with maps $f_i$ into each $X_i$ that are compatible with the $\varphi_{ij}$, there is a unique map $f$ through with all the $f_i$ factor, so that for all $i\in I$ the maps commute in
      \begin{displaymath}
        \xymatrix{
          X \ar[rr]^{p_i} && X_i  \\
          Z \ar@{-->}[u]^f \ar[urr]_{f_i} 
        }
      \end{displaymath}
      Note that this definition of a limit matches \cref{defn:projlim} except for allowing more general indexing.

      The next result shows that picking a suitable sub-system of a projective system will give the same limit object.
      Define some notation:
      \begin{itemize}
        \item A poset $A$ is called \emph{directed} if for all $a,b\in A$ there is some $c\in A$ such that~$c>a$ and~$c>b$.
        \item Two subsets of a poset $I,J\subseteq A$ are called \emph{cofinal} if for any $i\in I$ there is some $j\in J$ with $j\ge i$ and for any $j\in J$ there is some $i\in I$ with $i\ge j$.
          A subset $I\subset A$ is called \emph{cofinal in A} (or simply ``cofinal'') if $I$ and $A$ are cofinal as subsets of $A$.
      \end{itemize}
      \begin{thm}[Cofinal Limits are Naturally Isomorphic]
        \label{thm:cofinalisom}
        Let $I$ and $J$ be mutually cofinal \emph{directed} subsets of a poset $A$ and let a collection of objects $X_i$ for $i\in A$ be a projective system.
        Then there is a natural isomorphism
        \begin{align*}
          \lim_{i\in I} X_i \cong \lim_{j\in J} X_j
        \end{align*}
      \end{thm}
      \begin{proof}
        Write $X=\lim_I X_i$ and $X'=\lim_J X_j$.
        By definition, we make a map $\alpha:X\rightarrow X'$ by giving a compatible family of maps $X\rightarrow X_j$ for all $j\in J$.
        Using cofinality we may pick some~$i\in I$ with $i\ge j$ and make a map $X\rightarrow X_j$ by the composite
        \begin{displaymath}
          \xymatrix{
            X \ar[rr]^{p_i} \ar@{-->}[drr] && X_i \ar[d]^{\varphi_{ij}} \\
            && X_j
          }
        \end{displaymath}
        This definition is independent of $i$: for $i>i'>j$, use the directedness of $I$ to pick $\tilde{i}>i>i'$.
        The compatibility of the projections $p_i$ with the transition maps $\varphi_{ij}$ give a fully commutative diagram:
        \begin{displaymath}
          \xymatrix{
            X \ar@/^20pt/[rr]_{p_i} \ar@/^30pt/[rrr]^{p_{i'}} \ar[dr]^{p_{\tilde{i}}} && X_i \ar[ddr] & X_{i'} \ar[dd] \\
            & X_{\tilde{i}} \ar[ur] \ar[urr] \ar[drr] \\
            &&& X_j
          }
        \end{displaymath}
        leaving the transition maps $\varphi$ unlabeled.

        It remains to show that these induced maps are \emph{compatible}.
        Let $j>j'$ in $J$; for $i>j$ in~$I$ we have commutativity in
        \begin{displaymath}
          \xymatrix{
            X \ar[rr] && X_i \ar[d] \ar[dr] \\
            && X_j \ar[r] & X_{j'}
          }
        \end{displaymath}
        Thus the induced (composite) maps are compatible in the commutative diagram
        \begin{displaymath}
          \xymatrix{
            X \ar[rr] \ar@/^35pt/@{-->}[drrr]^{} \ar@/^10pt/@{-->}[drr] && X_i \ar[d] \ar[dr] \\
            && X_j \ar[r] & X_{j'}
          }
        \end{displaymath}
        Thus we have the map $X\rightarrow X'$.
        By a symmetric argument we have an induced map in the reverse direction.
        It remains to show that these maps are mutual inverses; the proof is similar to the proof that any two limits are isomorphic.
        So far we have a commutative diagram with induced dashed maps
        \begin{displaymath}
          \xymatrix{
            X \ar@/^/@{-->}[d] \ar@/^10pt/[rr] \ar@/^20pt/[rrr]^{}
            && X_i \ar[dr] & X_{i'} \\
            X' \ar@/^/@{-->}[u] \ar@/_10pt/[rr] \ar@/_20pt/[rrr]_{}
            && X_j \ar[ur] & X_{j'}
          }
        \end{displaymath}
        By construction, the composite $X\rightarrow X' \rightarrow X_{i'}$ is the same as projection $X\rightarrow X_{i'}$, so the map $X\rightarrow X' \rightarrow X$ will respect projections to the $X_i$, and so must be the identity.
        Symmetrically, we see that \emph{both} maps are isomorphisms.
      \end{proof}

      The natural isomorphism of cofinal limits is powerful, allowing easy understanding of limits with general indexing.
      This is a natural consequence of our characterizing definition of the limit---we find that facts about limits are true for general reasons.
      The previous section showed that (implicitly) countably indexed limits are unique and may be constructed as a subset of the product.
      These proofs also hold for projective systems indexed by a poset, for unsurprising reasons.

\end{document}
