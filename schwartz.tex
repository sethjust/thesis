\documentclass[thesis.tex]{subfiles}
\begin{document}
\onehalfspacing
% Change math display spacing here, so spacing commands don't reset it.
\abovedisplayskip=3pt
\belowdisplayskip=3pt
\abovedisplayshortskip=2pt
\belowdisplayshortskip=2pt

  \begin{savequote}
    ``May the Schwartz be with you!''
    \qauthor{--Yogurt, Space Balls}
  \end{savequote}
  \chapter{Topology on Schwartz Functions}
    \label{ch:topons}
    Our development of Schwartz functions and tempered distributions showed that these spaces have nice properties with regard to the Fourier transform.
    However, our discussion has mainly concerned itself with set-theoretic notions---that is, with the properties of elements of these spaces.
    We hope that these spaces are also nicely behaved with regard to \emph{topological} properties such as limits and continuity.
    Particularly, understanding \S as a topological space allows us to easily understand $\S^*$, as duals of topological spaces are well understood.

    \section{Spaces of Functions}
      We are interested in \emph{topology} on spaces of functions, with the goal of finding a topology for the space of distributions.
      Recall:
      \begin{defn}
        A \emph{topology} on a set $X$ is a set $\tau\subseteq2^X$ that satisfies:
        \begin{itemize}
          \item $\emptyset,X\in\tau$,
          \item $\tau$ is closed under arbitrary union,
          \item $\tau$ is closed under finite intersection.
        \end{itemize}

      Given a set $X$ with a \emph{metric} $d:X^2\rightarrow\R$, we take all sets of the form $\{x\in X:d(x,y)<\alpha\}$ for $y\in X$ and $\alpha\in\R$ as a basis for a set $\tau_d$, which is verifiably a topology on $X$.

      Finally, for a vectorspace $X$ over $\C$ with a \emph{norm} $|\cdot|:X\rightarrow\R$, we define a metric on $X$ by $d(x,y)=|x-y|$, and so can define a topology on $X$ by means of this metric.
      \end{defn}

      This final form of this definition is particularly useful in this context, as sets of functions \emph{naturally} form vectorspaces over \C, by defining operations pointwise:
      \begin{align*}
        (f+g)(x) &= f(x)+g(x)\\
        (a\cdot f)(x) &= a\cdot f(x)
      \end{align*}
      As such, we need only introduce a concept of \emph{size} on functions in order to get a topology on a set of functions (supposing it is closed under vectorspace operations).
      Importantly though, we care much more about the resulting \emph{topology} than we do about the norm, as topology is a more general (and relevant) concept.

      Consider the space $C^0_0(\R)$ of complex-valued, continuous functions vanishing at infinity, which is a vectorspace over $\C$.
      Give $C^0_0(\R)$ the ``sup norm'' and its associated metric:
      \begin{align*}
        |f|_{C^0_0} &= \sup_{x\in\R}|f(x)|\\
        d(f,g) &= |f-g|_{C^0_0}
      \end{align*}
      It is trivial that the sup norm is positive-definite and obeys the triangle inequality, by properties of the absolute value and sup.
      Thus the following completes the proof that sup norm is in fact a norm:
      \begin{claim}
        The sup norm on $C^0_0(\R)$ is always finite.
        \begin{proof}
          Let $f\in C^0_0(\R)$, so it vanishes at infinity.
          Thus, given $\varepsilon>0$ we have $N_\varepsilon>0$ such that $|x|>N\Rightarrow f(x)<\varepsilon$.
          This allows us to write
          \begin{align*}
            |f|_{C^0_0} &= \sup_{x\in\R}|f(x)|\\
            &= \max\left(\sup_{|x|<N_\varepsilon}|f(x)|,\varepsilon\right)
          \end{align*}
          which is finite by the extreme value theorem.
        \end{proof}
      \end{claim}

      The simplicity of the following proof illustrates the naturalness of the sup norm applied to $C^0_0(\R)$.
      \todo{Why is this important?}
      \begin{claim}
        For $x\in\R$, the evaluation functional $C^0_0(\R)\rightarrow\C$ given by $f\mapsto f(x)$ is continuous.
        \begin{proof}
          Given $\varepsilon > 0$, $d(f,g)<\varepsilon \Rightarrow |f(x)-g(x)|<\varepsilon$ for all $x$, showing continuity.
  %        Given $\varepsilon > 0$, $d(f,g)<\varepsilon \Rightarrow |f(x)-g(x)|<\varepsilon$ because $\forall x$ $|f(x)-g(x)|\le\sup_{y\in\R}|f(y)-g(y)|=d(f,g)$.
  %        Thus the evaluation functional is continuous.
        \end{proof}
      \end{claim}

      Finally, we see that we have managed to construct a nicely behaved topological space:
      \begin{thm}
        The space $C^0_0(\R)$ of continuous functions vanishing at infinity is \emph{complete}.
      \end{thm}
      \begin{proof}
        Consider a Cauchy sequence of $C^0_0$ functions $\{f_i\}$.
        First we will see that this sequence has a pointwise limit.

        Given $\varepsilon<0$ pick $N$ such that for $i,j>N$ we have $|f_i-f_j|<\varepsilon$.
        Then for any $x\in\R$, $|f_i(x)-f_j(x)|<\varepsilon$, and so the sequence of complex outputs $\{f_i(x)\}$ is Cauchy and has a limit $f(x)$.
        Thus the pointwise limit of $\{f_i\}$ is $f$.
       
        Now for $\varepsilon'>0$ take $j>N$ so that $|f_j(x)-f(x)|<\varepsilon'$.
        Then for $i>N$
        \begin{align*}
          |f_i(x)=f(x)|\le|f_i(x)-f_j(x)|+|f_j(x)-f(x)|<\varepsilon+\varepsilon'\text{.}
        \end{align*}
        Taking the limit as $\varepsilon'\rightarrow0$, we see $|f_i(x)-f(x)|<\varepsilon$ for \emph{all} $x$.
        Thus the convergence $\{f_i\}\rightarrow f$ is \emph{uniform} in $x$.
        Quoting the fact that the limit of continuous is continuous when convergence is uniform, we see that $f$ is continuous.\footnotemark

        It remains to show only that $f$ vanishes at infinity.
        Take $\varepsilon$ and $N$ as above, and fix some $i>N$.
        Then $|f_i(x)-f(x)|<\varepsilon$ for all $x$.
        Let $\varepsilon'>0$ and choose $M$ such that $|x|>M\Rightarrow |f_i(x)|<\varepsilon'$.
        Then
        \begin{align*}
          |f(x)|\le|f_i(x)-f(x)|+|f_i(x)|<\varepsilon+\varepsilon'\text{,}
        \end{align*}
        so for large $|x|$, we can make $f(x)$ arbitrarily small, and so $f\in C^0_0$.
      \end{proof}
      \footnotetext{This is easily verified by combining the inequalities above with the continuity of the $f_i$.}
      \begin{rmk}
        Thus, being complete with respect to the topology arising from a norm, $C^0_0(\R)$ is a \emph{Banach} space, by definition.
      \end{rmk}
      \begin{rmk}
        Note that this completeness is a consequence of our norm forcing sequences of functions to converge \emph{uniformly}.
        This is particularly important as pointwise limits of continuous functions need not be continuous!
      \end{rmk}

    \subsection{Spaces of differentiable functions}
      We now turn to the issue of \emph{differentiability}.
      Define the space 
      \begin{align*}
        C^k_0(\R) = \{f\in C^0_0(\R) : f'\in C^{k-1}_0(\R)\}
      \end{align*}
      Note that this definition requires that all $k+1$ derivatives $f,f',\ldots,f^{(k)}$ be $C^0_0$, and so they are \emph{all} continuous and vanish at infinity.
      Give $C^k_0(\R)$ the topology of the $C^k$-norm:
      \begin{align*}
        |f|_{C^k} = \sum_{0\le i\le k} |f^{(i)}|_{C^0} = \sum_{0\le i\le k} \sup_{x\in\R}|f^{(i)}(x)|
      \end{align*}
      \begin{rmk}
        The $C^k$ norm (or ``sum~sup'' norm) gives the same topology as the ``sup~sup'' norm:
        \begin{align*}
          |f|=\sup_{0\le i\le k}|f^{(i)}|_{C^0_0}=\sup_{0\le i\le k}\sup_{x\in\R}|f^{(i)}(x)|
        \end{align*}
        because for complex numbers $a_1,\ldots,a_k$
        \begin{align*}
          \sup_i a_i \le \sum_i a_i \le (k-1) \sup_i a_i\text{.}
        \end{align*}
        As such, the two can be used interchangeably, as we are concerned only with the resultant topology.
      \end{rmk}
      \begin{rmk}
        \todo{differentiation: $C^k_0\rightarrow C^{k-1}_0$ is continuous}
      \end{rmk}
      \begin{thm}
        $C^k_0(\R)$ is complete in the $C^k$-norm topology.
        \begin{proof}
          \todo{this}
        \end{proof}
      \end{thm}

    \subsection{Spaces of decaying functions}
      Now consider the issue of \emph{decay}.
      Recall that Schwartz functions are \emph{rapidly decaying}, so their product with any power of $x$ vanishes at infinity.
      Define
      \begin{align*}
        C^k_\ell(\R) = \{f\in C^k_0(\R):x^\ell f^{(i)}(x)\text{ vanishes at infinity for all } i\le k\}
      \end{align*}
      \begin{thm}
        $C^k_\ell(\R)$ is complete in the $C^k$-norm topology.
        \begin{proof}
          \todo{this}
        \end{proof}
      \end{thm}

    \section{Products and Limits of Topological Spaces}

\end{document}
