\documentclass[thesis.tex]{subfiles}
\begin{document}
\onehalfspacing
% Change math display spacing here, so spacing commands don't reset it.
\abovedisplayskip=3pt
\belowdisplayskip=3pt
\abovedisplayshortskip=2pt
\belowdisplayshortskip=2pt

  \setcounter{chapter}{1}
  \begin{savequote}
    ``May the Schwartz be with you!''
    \qauthor{--Yogurt, Space Balls}
  \end{savequote}
  \chapter{Topology on Schwartz Functions}
    \label{ch:topons}
    Our development of Schwartz functions and tempered distributions showed that these spaces have good properties with regard to the Fourier transform.
    However, our discussion has mainly concerned itself with set-theoretic notions---that is, with the properties of elements of these spaces.
    We hope that these spaces are also nicely behaved with regard to \emph{topological} properties such as continuity and limits.
    In fact, understanding \S as a topological space allows us to easily understand $\SS$ as a topological space without getting caught up in details.
    This understanding allows us to carefully discuss approximating tempered distributions with other functions; we will eventually see that any distribution is a limit (in the appropriate topology) of Schwartz functions.

%    Topological understanding also offers great power for .
%    By working in the category of continuous maps between topological spaces we are able to avoid much of the clutter of a classical working environment.

    \section{Spaces of Functions}
      \label{sec:spcoffunc}
      We are interested in topology on spaces of functions, with the goal of finding a topology for the space of distributions.
      Recall:
      \begin{defn}
        A \emph{topological space} is a set $X$ and a set $\tau\subseteq2^X$ that satisfies:
        \begin{itemize}
          \item The empty set $\emptyset$ and the whole space $X$ are elements of $\tau$,
          \item $\tau$ is closed under arbitrary union,
          \item $\tau$ is closed under finite intersection.
        \end{itemize}
        We call $\tau$ a \emph{topology on $X$}.

        Given a set $S$ of subsets of $X$ that contains $\emptyset$ and $X$, generate a topology $\tau_S$ on $X$ by completing $S$ with respect to union and finite intersection (that is, taking all possible unions and finite intersections of sets in $S$ repeatedly until no new open sets are generated).
        $S$ is called a \emph{generating set} or \emph{sub-basis} of the topology $\tau_S$.
        If $S$ is closed under finite intersection (so $\tau_S$ is generated by unions of elements of $S$), $S$ is called a \emph{basis} of $\tau_S$.

        Given a set $X$ with a metric\footnotemark $d:X^2\rightarrow\R_{\ge 0}$, we take all sets of the form 
        \begin{align*}
          \{x\in X:d(x,y)<\alpha\}
        \end{align*}
        for $y\in X$ and $\alpha\in\R_{\ge0}\cup\{\infty\}$ as a generating set for a topology $\tau_d$ on $X$.
      \end{defn}
      \footnotetext{Recall that a metric satisfies (a) $d(x,y)=0\Leftrightarrow x=y$, (b) $d(x,y)=d(y,x)$, and (c) $d(x,z)\le d(x,y)+d(y,z)$.}

      The category of general topological spaces is too general to usefully apply to our context.
      By moving to a category that offers additional structure (such as the category of metric spaces, where we may use intuition about size) we are able to more easily grasp the topologies we work with.
      We hope that we can find some such structure on spaces of functions that allow us to easily understand them as topological spaces.
      Happily, properly chosen sets of functions naturally form vectorspaces over \C, by defining operations pointwise:
      \begin{align*}
        (f+g)(x) &= f(x)+g(x)\\
        (a\cdot f)(x) &= a\cdot f(x) \text{.}
      \end{align*}
      This vectorspace structure lets us easily construct topologies by providing a concept of the \emph{size} of functions.
      \begin{defn}
        A \emph{norm} on a complex vectorspace $X$ is a function
        \begin{align*}
          |\cdot|_X:X\longrightarrow\R_{\ge0}
        \end{align*}
        that satisfies for all $x,y\in X$, $a\in\C$:
        \begin{itemize}
          \item positive scalability: $|ax|_X=|a|\cdot|x|_X$;
          \item triangle inequality: $|x+y|_X\le|x|_X+|y|_X$;
          \item separation of points: $|x|_X=0\Longleftrightarrow x=0_X$.
        \end{itemize}
        Notationally, some subscripts are dropped and we distinguish between a norm and the complex absolute-value function by context.
      \end{defn}

      Now suppose that $X$ is a vectorspace over $\C$ with a norm $|\cdot|$.
      We define a metric on $X$ by $d(x,y)=|x-y|$, and so can define a topology on $X$ by means of this metric.
      Importantly though, distinct norms can give rise to the same topology (which we will see later).
      Thus the norm we use to induce a topology is not itself important, but rather a tool that lets us easily find the topology which we are interested in.

      Topologies arising from norms naturally respect vectorspace operations:
      \begin{claim}
        The addition and scalar multiplication maps 
        \begin{align*}
          +:X\times X\rightarrow X
          \qquad\text{and}\qquad
          \cdot:\C\times X\rightarrow X
        \end{align*}
        are (jointly) continuous.\footnotemark
      \end{claim}
      \footnotetext{
        Recall that a map $f:\prod_{1\le i\le n}X_i\rightarrow K$ is jointly continuous if for any set $Z$ with a family of continuous maps $g_i:Z\rightarrow X_i$ the map $Z\rightarrow K$ given by $z\mapsto f(g_1(z),g_2(z),\ldots,f_n(z))$ is continuous.
      }
      \begin{proof}
        Consider some space $Z$ with continuous maps $g_1,g_2:Z\rightarrow X$.
        We hope the map $g:Z\rightarrow X$ given by $z\mapsto g_1(z)+g_2(z)$ is continuous.
        Let $\{z_i\}$ be a sequence converging to some $z\in Z$; to show that $\{g(z_i)\}\rightarrow g(z)$ it suffices to show $\{|g(z_i)-g(z)|\}\rightarrow0$.
        See that 
        \begin{align*}
          |g(z_i)-g(z)|
          &\le |g_1(z_i) - g_1(z)| + |g_2(z_i) - g_2(z)|\text{.}
        \end{align*}
        Because the $g_i$ are continuous each of the right-hand summands approaches zero.
        Thus $g$ is continuous, so the addition map is jointly continuous.

        Now consider a pair of continuous maps $g_1:Z\rightarrow\C$ and $g_2:Z\rightarrow X$ and the corresponding $g:Z\rightarrow X$ given by $z\mapsto g_1(z)\cdot g_2(z)$.
        As above, calculate
        \begin{align*}
          |g(z_i)-g(z)|
          &= |g_1(z_i)\cdot g_2(z_i) - g_1(z)\cdot g_2(z)|
          \\&= |(g_1(z_i)-g_1(z)+g_1(z))\cdot g_2(z_i) - g_1(z)\cdot g_2(z)|
          \\&= |(g_1(z_i)-g_1(z))\cdot g_2(z_i) + g_1(z)\cdot(g_2(z_i)-g_2(z))|
          \\&\le |g_1(z_i)-g_1(z)|\cdot|g_2(z_i)| + |g_1(z)|\cdot|g_2(z_i)-g_2(z)| \text{.}
        \end{align*}
        Again, the continuity of the $g_i$ force both summands to zero, so the multiplication map is jointly continuous.
      \end{proof}

      We call any vectorspace with a topology in which this joint continuity holds a \emph{topological vectorspace} (distinguishing them from vectorspaces with a less-well-behaved topology).
      Thus we see that any normed vectorspace is naturally a topological vectorspace.

      It is also worth noting that certain maps on topological vectorspaces are easy to work with.
      By the definition of continuity, a map from a topological vector space into some topological set $f:X\rightarrow Y$ is continuous iff for all $a\in X$
      \begin{align*}
        \lim_{x\rightarrow 0} f(a+x) = f(a) \text{.}
      \end{align*}
      However, when $f$ is linear, we may simplify this condition.
      For any $a\in X$ the following are equivalent:
      \begin{gather*}
        \lim_{x\rightarrow 0} f(a+x) = f(a) \\
        \lim_{x\rightarrow 0} f(a)+f(x) = f(a) \\
        \lim_{x\rightarrow 0} f(x) = 0\text{.}
      \end{gather*}
      Thus a linear map on a topological vectorspace is continuous anywhere if and only if it is continuous at zero.
      Intuitively this is because to topology anywhere is just a translated copy of the topology around zero, and linear maps respect translation.

      \subsection{A space of continuous, vanishing functions}

      We have see that we need only introduce a concept of size on functions (via a norm) in order to get a topology on a set of functions (supposing it is closed under vectorspace operations).

      Now consider the space $\Coo(\R)$ of complex-valued, continuous functions vanishing at infinity, which is a vectorspace over $\C$.
      Notationally, take $\Coo=\Coo(\R)$ for clarity, as we are concerned only with functions on the line.
      Give $\Coo$ the ``sup norm'' and its associated metric:
      \begin{gather*}
        |f|_\Coo = \sup_{x\in\R}|f(x)|\\
        d(f,g) = |f-g|_\Coo \text{.}
      \end{gather*}
      It is trivial that the sup norm separates points and obeys positive scalability and the triangle inequality, by properties of the absolute value and sup.
      Thus the following completes the proof that sup norm is in fact a norm:
      \begin{claim}
        The sup norm on $\Coo$ is always finite.
        \begin{proof}
          Let $f\in \Coo$, so it vanishes at infinity.
          Thus, given $\varepsilon>0$ we have $N>0$ such that $|x|>N\Rightarrow f(x)<\varepsilon$.
          This allows us to write
          \begin{align*}
            |f|_\Coo
            = \sup_{x\in\R}|f(x)|
            = \max\left(\varepsilon,\,\sup_{|x|<N}|f(x)|\right)
          \end{align*}
          which is finite by the extreme value theorem.
        \end{proof}
      \end{claim}
      We thus see that \Coo is a topological vectorspace with topology induced by the sup norm.
      This structure allows us to easily interact with interesting properties of the space.
      For example:
      \begin{claim}
        \label{claim:c00evalcts}
        For $x\in\R$, the evaluation functional $\Coo(\R)\rightarrow\C$ given by $f\mapsto f(x)$ is continuous.
        \begin{proof}
          See that $|f(x)| \le \sup_{x\in\R} |f(x)| = |f|_\Coo$, so the evaluation functional is continuous.
        \end{proof}
      \end{claim}
      \begin{rmk}
        \label{cor:limitworks}
        While this proof seems silly in its simplicity, or even completely obvious, we have in fact proved the topology on \Coo agrees nicely with an intuitive sense of the structure of functions: limits of functions in \Coo agree with the normal (pointwise) limit of functions.
        That is, supposing a sequence $\{f_i\}$ in \Coo has a limit $f\in\Coo$, we have for any $x\in\R$
        \begin{align*}
          (\lim_i f_i)(x) &= \lim_i f_i(x)\text{.}
        \end{align*}
      \end{rmk}

      The result above requires that we assume the sequence $\{f_i\}$ has a limit in \Coo.
      We hope that this is always that case when the sequence is well-behaved, or \emph{Cauchy}.
      \begin{defn}
        A sequence $\{x_i\}$ in a space $X$ with a topology arising from a metric $d$ is called Cauchy when elements of the sequence become arbitrarily close to one another.
        That is the sequence is Cauchy iff for any $\varepsilon>0$ there is some $N>0$ such that for all $i,j>N$ we have $d(x_i,x_j)<\varepsilon$.

        If every such sequence in $X$ converges to a limit in $X$, we say \emph{$X$ is complete in its topology}, or simply \emph{complete}.
        A complete topological vectorspace is a \emph{Banach space}.
      \end{defn}

      \begin{thm}
        \label{thm:c00complete}
        The space $\Coo(\R)$ of continuous functions vanishing at infinity is complete.
      \end{thm}
      \begin{proof}
        Consider a Cauchy sequence of \Coo functions $\{f_i\}$.
        First, see that this sequence has a pointwise limit.
        For any $i,j$ and $x\in\R$ see that $|f_i(x)-f_j(x)| \le |f_i-f_j|_\Coo$, so the sequence of complex outputs $\{f_i(x)\}$ is Cauchy and has a limit $f(x)$.
        By \cref{cor:limitworks} we may unambiguously write $f=\lim_i f_i$.

        The limit $f$ must be continuous, by definition of \Coo.
        Fix some $a\in\R$, so letting $x$ vary
        \begin{align*}
          |f(x)-f(a)|
          &\le |f(x)-f_i(x)|+|f_i(x)-f_i(a)|+|f_i(a)-f(a)|
        \end{align*}
        by the triangle inequality.
        The outer terms approach zero as $i\rightarrow\infty$, and the middle term vanishes as $x\rightarrow a$.
        Thus $f$ is continuous.

        Finally, we hope that $f$ vanishes at infinity.
        Applying the triangle inequality for any $x$ and $i$, see
        \begin{align*}
          |f(x)| \le |f(x)-f_i(x)| + |f_i(x)|
          \underset{i, |x|\rightarrow\infty}{\longrightarrow} 0\text{.}
        \end{align*}
        Thus $f$ vanishes at infinity, so $\lim_i f_i = f\in\Coo$.
      \end{proof}
      \begin{cor}
        Being complete with respect to the topology arising from a norm, \Coo is a Banach space, by definition.
      \end{cor}
      \begin{rmk}
        Note that this completeness is a consequence of our norm forcing sequences of functions to converge \emph{uniformly}.
        This is particularly important as pointwise limits of continuous functions need not be continuous.
        For example, the function $f_n$ is continuous for all $n$, but the limiting function is discontinuous at zero:
        \begin{align*}
          &f_n(x)=
          \begin{cases}
            -1 &\text{if } x<-\frac1n\\
            nx &\text{if } |x|<\frac1n\\
            1 &\text{if } x>\frac1n
          \end{cases}
          &\lim_nf_n(x)=
          \begin{cases}
            -1 &\text{if } x<0\\
            0 &\text{if } x=0\\
            1 &\text{if } x>0
          \end{cases}
        \end{align*}
      \end{rmk}

    \subsection{Spaces of differentiable functions}
      Recall that Schwartz functions are infinitely continuously differentiable, while functions in \Coo are only guaranteed to themselves be continuous, while their derivatives need not exist, let alone be continuous.
      We thus hope that we can find a space of functions with any finite number of derivatives. 
      For $k>0$ define the space 
      \begin{align*}
        \Cko(\R) = \{f\in \Coo(\R) : f'\in \Ckmo(\R)\} \text{.}
      \end{align*}
      This definition requires that all $k+1$ derivatives $f,f',\ldots,f^{(k)}$ be \Coo, and so they are \emph{all} continuous and vanish at infinity.
      Give $\Cko(\R)$ the topology of the \Cko-norm:
      \begin{align*}
        |f|_\Cko = \sum_{0\le i\le k} |f^{(i)}|_\Coo = \sum_{0\le i\le k} \sup_{x\in\R}|f^{(i)}(x)| \text{.}
      \end{align*}
      Note that for $k=0$, this is exactly the sup norm, and so the notation \Cko is consistent with writing \Coo.
      \begin{rmk}
        The \Cko norm (or ``sum~sup'' norm) gives the same topology as the ``sup~sup'' norm:
        \begin{align*}
          |f|=\sup_{0\le i\le k}|f^{(i)}|_\Coo=\sup_{0\le i\le k}\sup_{x\in\R}|f^{(i)}(x)|
        \end{align*}
        because for complex numbers $a_1,\ldots,a_k$
        \begin{align*}
          \sup_i |a_i| \le \sum_i |a_i| \le k\cdot\sup_i |a_i|\text{.}
        \end{align*}
        Thus the two can be used interchangeably, as we are concerned only with the resultant topology.
      \end{rmk}
      \begin{claim}
        \label{claim:diffcontCk0}
        The differentiation map $D:\Cko\rightarrow\Ckmo$ is continuous.
        \begin{proof}
          Let $f\in \Cko$ and calculate
          \begin{align*}
            |Df|_\Ckmo
            = \sum_{1\le i\le k} |f^{(i)}|_\Coo
            \le |f|_\Cko \text{.}
          \end{align*}
          Thus when $f$ is small in \Cko, $Df$ is small in \Ckmo.
        \end{proof}
      \end{claim}
      \begin{thm}
        \label{thm:ck0complete}
        $\Cko(\R)$ is complete in its topology.
      \end{thm}
      \begin{proof}
        Consider the case $k=1$; this will extend inductively to all $k$.
        Let $\{f_i\}$ be Cauchy in $C^1_0$.
        Because both sequences $\{f_i\}$ and $\{f_i'\}$ are in \Coo, by definition of the \Coo (sup) norm and $C^1_0$ norm, each is Cauchy in \Coo.
        Thus by \cref{thm:c00complete} each sequence converges uniformly pointwise, so we can write:
        \begin{align*}
          &f(x)=\lim_i f_i(x) &g(x)=\lim_i f_i'(x) \text{.}
        \end{align*}
        See that these expressions are unambiguous by \cref{cor:limitworks}.

        We now hope to show that $f$ is differentiable, and so an element of $C^1_0$.
        Since each $f_i$ is continuous and differentiable, write
        \begin{align*}
          f_i(x)-f_i(a) = \int_a^x f_i'(t)\,dt \text{.}
        \end{align*}
        Taking the limit, and exchanging limit and integral by the Lebesgue Dominated Convergence Theorem we have that
        \begin{align*}
          \lim_i \left( f_i(x)-f_i(a) \right) &= \lim_i\int_a^x f_i'(t)\,dt
          = \int_a^x\lim_if_i'(t)\,dt
        \end{align*}
        and so we see 
        \begin{align*}
          f(x)-f(a) &= \int_a^x g(t)\,dt \text{.}
        \end{align*}
        Thus, by the fundamental theorem of calculus, $f'=g$, and so $f\in C^1_0$.
        Thus $C^1_0$ is complete in its topology, proving the base case.

        Now suppose that $\{f_i\}$ is Cauchy in \Cko for $k>1$, and that \Ckmo is complete in the topology induced by its norm.
        Again, both $\{f_i\}$ and $\{f_i'\}$ are Cauchy in \Coo, so letting $f$ and $g$ be the limits of the sequences (as above), we see that $f'=g$.
        Because $\{f_i'\}$ is also Cauchy in \Ckmo, by the inductive hypothesis we have $g\in\Ckmo$, and so $f\in\Cko$, finishing the proof.
      \end{proof}

    \subsection{Spaces of decaying functions}
      \label{sec:decayfunc}
      Having constructed a topological space of functions with any finite number of derivatives, we now consider the issue of decay.
      Recall that Schwartz functions are \emph{rapidly decaying}, so their product with any power of $x$ vanishes at infinity.
      For $\ell\ge0$ define (for now)
      \begin{align*}
        \Cl(\R) = \{f\in \Coo(\R):(1+x^2)^\ell f(x)\in \Coo(\R)\}
      \end{align*}
      with the norm:
      \begin{align*}
        |f|_\Cl = |(1+x^2)^\ell f|_{\Coo} = \sup_{x\in\R} (1+x^2)^\ell|f(x)| \text{.}
      \end{align*}
      Note that for $\ell=0$ we have $\Cl=\Coo$, and their respective norms agree.

      The definition of \Cl is more complicated that we might hope.
      Ideally the $(1+x^2)$ would be an $x$, but importantly $(1+x^2)^\ell\ge1$ for all $x$ and $\ell$.
      As the next proof shows, this detail is vital.
      Replacing $(1+x^2)^\ell$ with $x^\ell$ in the definition would allow our functions to blow up near the origin (where $x^\ell$ is very small), which would allow us to construct a Cauchy sequence that does not converge in \Cl.
      \begin{thm}
        \label{thm:c0lcomplete}
        $\Cl(\R)$ is complete in its topology.
      \end{thm}
      \begin{proof}
        Take a Cauchy sequence of \Cl functions $\{f_i\}$.
        Because each $f_i\in \Coo$ we want to say that $\{f_i\}$ has a pointwise limit $f$ that itself is $\Coo$, as in the proof of \cref{thm:c00complete}.
        This holds only if $\{f_i\}$ is Cauchy in \Coo, but for any $i$ and $j$ 
        \begin{align*}
          |f_i-f_j|_\Coo
          = \sup_{x\in\R}|f_i(x)-f_j(x)|
          \le \sup_{x\in\R}(1+x^2)^\ell|f_i(x)-f_j(x)|
          =|f_i-f_j|_\Cl
        \end{align*}
        and so we may write $f=\lim_i f_i$ in \Coo.
        Thus it remains only to show that $(1+x^2)^\ell f(x)$~is~\Coo.
        
        By hypothesis the sequence $\{(1+x^2)^\ell f_i\}$ in \Coo is Cauchy.
        Then it has a limit in \Coo, say $g$.
        We hope that $g(x)=(1+x^2)^\ell f(x)$.
        Fix some $x\in\R$ and see
        \begin{align*}
          |g(x) - (1+x^2)^\ell f(x)| &\le |(1+x^2)^\ell f_i(x) - g(x)|+(1+x^2)^\ell|f_i(x)-f(x)| \text{.}
        \end{align*}
        Because $x$ is fixed, the right hand expression becomes arbitrarily small as $i$ grows and both limits converge.
        Thus $(1+x^2)^\ell f$ is \Coo, and so $f\in\Cl$.
      \end{proof}

      With the basic cases proved, we now hope to define a Banach space that is parameterized both by smoothness and decay properties.
      \begin{defn}
        The space of $k$-times differentiable functions decaying faster than $(1+x^2)^\ell$ is
        \begin{align*}
          \Ckl(\R) = \{f\in \Cko(\R):(1+x^2)^\ell f^{(i)}(x)\in C^{k-i}_0\text{ for all } i\le k\}
        \end{align*}
        with the norm
        \begin{align*}
          |f|_\Ckl = \sum_{0\le i\le k} \left|(1+x^2)^\ell f^{(i)}\right|_\Coo = \sum_{0\le i\le k} \sup_{x\in\R} \left|(1+x^2)^\ell f^{(i)}(x)\right| \text{.}
        \end{align*}
        Note that for $k=0$, $\Ckl=\Cl$, and their respective norms agree.
      \end{defn}

      \begin{claim}
        \label{claim:diffcontCkl}
        The differentiation map $D:\Ckl\rightarrow\Ckml$ is continuous.
        \begin{proof}
          Let $f\in\Ckl$.
          Then
          \begin{align*}
            |Df|_\Ckml
            = \sum_{1\le i\le k} \left| (1+x^2)^\ell f^{(i)} \right|_\Coo
            \le \sum_{0\le i\le k} \left| (1+x^2)^\ell f^{(i)} \right|_\Coo
            = |f|_\Ckl\text{.}
          \end{align*}
        \end{proof}
      \end{claim}

      \begin{claim}
        \label{claim:cklevalects}
        For any $x\in\R$ the evaluation functional $\Ckl\rightarrow\C$ given by $ \varphi\mapsto\varphi(x) $ is continuous.
        \begin{proof}
          See that, as in \cref{claim:c00evalcts}, for a sequence $\{\varphi_i\}\rightarrow0$ in \Ckl
          \begin{align*}
            |\varphi(x)| \le \sup_{x\in\R}|\varphi(x)| \le |\varphi|_\Ckl\text{.}
          \end{align*}
        \end{proof}
      \end{claim}

      \begin{thm}
        \label{thm:cklcomplete}
        $\Ckl(\R)$ is complete in its topology.
      \end{thm}
      \begin{proof}
        Again consider the case of $k=1$; proof will extend inductively to all $k$.
        This proof is similar to \cref{thm:ck0complete}, with the added complication(s) of proving decay along with differentiability.

        Take $\{f_i\}$ to be Cauchy in $C^1_\ell$.
        \begin{align*}
          |f_i-f_j|_{C^1_0}
          &= \sum_{0\le h\le 1}\sup_{x\in\R}|f_i^{(h)}(x)-f_j^{(h)}(x)|
          \\&\le \sum_{0\le h\le 1}\sup_{x\in\R}(1+x^2)^\ell|f_i^{(h)}(x)-f_j^{(h)}(x)|
          \\&= |f_i -f_j|_{C^1_\ell} \text{.}
        \end{align*}
        Thus $\{f_i\}$ is Cauchy in $C^1_0$.
        From \cref{thm:ck0complete} see that $\{f_i\}\rightarrow f$ in $C^1_0$ and $\{f_i'\}\rightarrow g$ in \Coo with $f'=g$.

        It remains to show that $(1+x^2)^\ell f\in C^1_0$ and $(1+x^2)^\ell g\in\Coo$.
        If $\{f_i\}$ is Cauchy in \Cl then $(1+x^2)^\ell f\in\Coo$ by \cref{thm:c0lcomplete}.
        Similarly, if $\{f_i'\}$ is Cauchy in \Cl, we'll see that $(1+x^2)^\ell g\in\Coo$.
        Thus proving the two sequences are Cauchy will show that $(1+x^2)^\ell f\in C^1_0$.
        Using the inequality $x(1+x^2)^{\ell-1} \le (1+x^2)^\ell$, calculate
        \begin{align*}
          |f_i-f_j|_\Cl
          = \sup_{x\in\R}(1+x^2)^\ell|f_i(x)-f_j(x)|
          \le |f_i -f_j|_{C^1_\ell} \text{.}
        \end{align*}
        The same inequality with first derivatives allows us to see that both sequences are indeed Cauchy in \Cl, and so $(1+x^2)^\ell f\in C^1_0$ and $(1+x^2)^\ell g\in\Coo$.

        Finally we calculate
        \begin{align*}
          \frac{d}{dx}(1+x^2)^\ell f(x)
          &= (1+x^2)^\ell f'(x) + \ell(1+x^2)^{\ell-1}2xf(x)
          \le (1+x^2)^\ell (g(x) + 2\ell f(x)) \text{.}
        \end{align*}
        We have just shown that both summands are \Coo, and so vanish at infinity.
        Thus, by the definition of $C^1_0$, $f\in C^1_0$, proving the theorem for $k=1$.

        Now take $\{f_i\}$ to be Cauchy in \Ckl for $k>1$, and suppose that \Ckml is complete.
        Then by the argument above see that $f=\lim_i f_i$ is $C^1_\ell$, $g=\lim_i f_i'$ is \Cl, and $f'=g$.
        Calculate
        \begin{align*}
          |f_i'-f_j'|_\Ckml
          = \sum_{1\le h\le k} (1+x^2)^\ell\left|f_i^{(h)}-f_j^{(h)}\right|
          \le |f_i-f_j|_\Ckl
        \end{align*}
        and so $g\in\Ckml$.
        Thus $f\in\Ckl$ so \Ckl is complete in its topology, and thus Banach.
      \end{proof}

    \section{\S as a Topological Limit}
      
      We have just see in \cref{thm:cklcomplete} that for all $k,\ell>0$, the space \Ckl of $k$-times differentiable functions decaying faster than $(1+x^2)^\ell$ is a Banach space (complete normed vectorspace).
      Because the space of Schwartz functions is made up \emph{infinitely} differentiable functions with infinitely fast decay, it seems natural that \S is the limit of these Banach spaces.
      \cref{ch:topprod} defines the \emph{projective limit} of topological spaces and shows that these limits are themselves topological spaces.
      We hope to show that this doubly-indexed limit is still topologically well-behaved and converges to \S.
      With this accomplished we will be able to completely understand \S as a topological space, and thus understand the tempered distributions \SS as a topological dual space.

      To take the limit of the \Ckl we must first define transition maps $\varphi^{k,k-1}_\ell$ and $\phi^k_{\ell,\ell-1}$:
      \begin{displaymath}
        \xymatrix{
          &&& \cdots \ar[d]_{\phi^1_{43}}\\
          && \cdots \ar[d]_{\phi^2_{32}} \ar[r]^{\varphi^{21}_3} & C^1_3 \ar[d]_{\phi^1_{32}}\\
          & \cdots \ar[d]_{\phi^3_{21}} \ar[r]^{\varphi^{32}_2} & C^2_2 \ar[d]_{\phi^2_{21}} \ar[r]^{\varphi^{21}_2} & C^1_2 \ar[d]_{\phi^1_{21}}\\
          \cdots \ar[r]^{\varphi^{43}_1} & C^3_1 \ar[r]^{\varphi^{32}_1} & C^2_1 \ar[r]^{\varphi^{21}_1} & C^1_1
        }
      \end{displaymath}
      To meet the limit definition these maps must be continuous and compatible.
      A natural idea is to make these maps inclusions.
      See that by definition $\Ckl\subset\Ckml$ (in the sense of sets), and similarly $\Ckl\subset C^k_{\ell-1}$.
      \begin{claim}
        \label{claim:inclcont}
        The inclusions $\Ckl\rightarrow \Ckml$ and $\Ckl\rightarrow C^k_{\ell-1}$ are continuous and compatible.
        Thus, letting the maps $\varphi^{k,k-1}_\ell$ and $\phi^k_{\ell,\ell-1}$ be these inclusions, the diagram above is a commutative diagram of continuous maps.
      \end{claim}
      \begin{proof}
        First see that by definition of set inclusion, the diagram must be commutative.
        Because any $f$ in some \Ckl is an element of every \Cij with $i\le k$ and $j\le\ell$ by definition, any composition of inclusions $\Ckl\rightarrow\Cij$ will agree.

        It now remains to show that the mappings are continuous.
        First consider $\varphi^{k,k-1}_\ell$ (the horizontal direction).
        Let $f$ be a function in \Ckl.
        \begin{align*}
          |\varphi^{k,k-1}_\ell(f)|_\Ckml
          = \sum_{0\le i \le k-1} \left|(1+x^2)^\ell f^{(i)}\right|_\Coo
          \le \sum_{0\le i \le k} \left|(1+x^2)^\ell f^{(i)}\right|_\Coo
          = |f|_\Ckl
        \end{align*}
        and so $\varphi^{k,k-1}_\ell$ is continuous.

        Now consider $\phi^k_{\ell,\ell-1}$ (the vertical direction).
        \begin{align*}
          |\phi^k_{\ell,\ell-1}(f)|_{C^k_{\ell-1}}
          = \sum_{0\le i \le k} \left|(1+x^2)^{\ell-1} f^{(i)}\right|_\Coo
          \le \sum_{0\le i \le k} \left|(1+x^2)^{\ell} f^{(i)}\right|_\Coo
          = |f|_\Ckl
        \end{align*}
        and so $\phi^k_{\ell,\ell-1}$ is continuous.
      \end{proof}
      \begin{rmk}
        This shows that, as a set, the limit of the \Ckl is a nested intersection:
        \begin{align*}
          \lim_{k,\ell}\Ckl = \bigcap_{k,\ell=0}^\infty \Ckl\text{.}
        \end{align*}
        Because any Schwartz functions is by definition an element of every \Ckl, as sets $\S\subseteq\lim_{k,\ell}\Ckl$.
      \end{rmk}

      To understand the limit as a topological space, we must do slightly more work.
      Give the index set $A=\{(k,\ell):k,\ell>0\}$ a partial ordering
      \begin{align*}
        (k,\ell) > (i,j) \quad\text{when $k\ge i$, $\ell\ge j$ and $(k,\ell) \ne (i,j)$.} 
      \end{align*}
      Indexing the \Ckl naturally by $A$, we have a projective system with inclusions as transition maps.
      Thus, by \cref{claim:limitexists} the \Ckl have a projective limit.

      \begin{thm}
        The space of Schwartz Functions is the limit of the projective system formed by the \Ckl.
      \end{thm}
      \begin{proof}
        See that any Schwartz function is by definition an element of every \Ckl, and so we take the maps $\varphi^k_\ell:\S\rightarrow\Ckl$ to be inclusions.
        These maps are visibly compatible with the transition maps of the projective system.

        Now consider an arbitrary space $Z$ with a family of maps $f^k_\ell:Z\rightarrow\Ckl$ compatible with the transition maps.
        We must show there is a unique $f:Z\rightarrow\S$ such that for all $k,\ell$, $\varphi^k_\ell\circ f = f^k_\ell$.
        For some index $(i,j)$, pick some $(k,\ell)>(i,j)$ so that we have (by hypothesis) the commutative diagram:
        \begin{displaymath}
          \xymatrix{
            \S \ar@/^20pt/[rr]_{\varphi^k_\ell} \ar@/^30pt/[rrr]^{\varphi^i_j} && \Ckl \ar[r] & \Cij \\
            & Z \ar[ur]^{f^k_\ell} \ar[urr]_{f^i_j}
          }
        \end{displaymath}
        For any $z\in Z$, $f^i_j(z)=f^k_\ell(z)$ because the horizontal (transition) map is inclusion.
        Because this holds for arbitrary $(k,\ell)$, $f^i_j(z)$ has arbitrarily many derivatives and decays arbitrarily quickly, and thus is Schwartz.
        Thus for all $(k,\ell)$, $f^k_\ell(z)=s$ for some fixed $s\in\S$; we define $f(z)$ to be exactly this $s$.
        This $f$ is unique because if $g(z)\ne f(z)$ we have that $\varphi^k_\ell(g(z))\ne\varphi^k_\ell(f(z))$ by the nature of inclusion, and so the $f^k_\ell$ do not factor through $g$.
        Thus $\S=\lim_{k,\ell}\Ckl$.
      \end{proof}
      \begin{rmk}
        The language of projective limits somewhat obscures the intuition that $\S$ is exactly the intersection of the \Ckl.
        Because the limit is an intersection the transition maps are inclusions, and so picking compatible maps into each limitand is the same as picking a map into the limit.
        Importantly, by setting the proof in a topological context, the description of \S as a limit allows us to easily describe its topology, while the set-theoretic construction offers no help.
      \end{rmk}
      \begin{defn}
        A \emph{Fr\'echet} space is a topological space that is the projective limit of Banach spaces.
        For example, as the projective limit of Banach spaces, \S is a Fr\'echet space by definition.

        Equivalently (though we will not use this definition), a Fr\'echet space is a complete topological vector space whose topology is induced by a countable family of semi-norms (norms that need not separate points).
        See \citet[\s 14]{functionsoncircles} for a full treatment of the various definitions of a Fr\'echet space.
      \end{defn}

      Now consider the limit topology on \S.
      For $i=(k,\ell)$ in the index set $A$, write $\varphi^k_\ell = \varphi_i$ and $\Ckl=X_i$ to reduce clutter.
      Open subsets of \S are abitrary unions and finite intersections of sets with the form $\varphi_i^{-1}(u_i)$ where $u_i$ is open in $X_i$.
      Because union and intersection pass through taking inverse images, see that the topology on \S is in fact generated by sets of the form $\varphi_i^{-1}(\{f\in X_i:|f-f_0|_{X_i}<r\})$ for $r\in\R_{>0}$ and $f_0\in X_i$.
      But because $\varphi_i$ is inclusion, taking its inverse image is exactly intersection with \S.
      Thus a subset $U$ of \S is open if and only if for each point $u\in U$ there are finitely many $j_i\in A$ and open $V_i\subseteq X_{j_i}$ so that for all~$i$ we have $u\in V_i$ and $V_i\cap\S\subseteq U$.
      Comparing this expression with the seminorm definition of a Fr\'echet space shows that the \Ckl norms form a family of seminorms on \S that give the limit topology.
%      \todo{Show that these are in fact \emph{semi}-norms, and \emph{not} proper norms.}

    \section{Consequences of Topology on \S}
      The following results arise from the limit topology on \S.
      We see that the topology gracefully interacts with natural operations on functions.

      \begin{thm}
        Differentiation $D:\S\rightarrow\S$ is continuous.
      \end{thm}
      \begin{proof}
        Recall that the differentiation map $D:\Ckl\rightarrow\Ckml$ and the inclusion $\Ckl\rightarrow\Cij$ (for $(k,\ell)>(i,j)$ in $A$) are continuous (\cref{claim:diffcontCkl,claim:inclcont}).
         Composing such functions for any $(k,\ell)>(i,j)$ gives a continuous map $D:\Ckl\rightarrow\Cij$ that takes a function in one space to its derivative in another.
        Thus we have the commutative diagram of continuous maps
        \begin{displaymath}
          \xymatrix{
            \S \ar@/^20pt/[rr]_{\varphi^k_\ell} \ar@/^30pt/[rrr]^{\varphi^i_j} & \cdots \ar[r] & \Ckl \ar[r] & \Cij \ar[r] & \cdots \\
            \S \ar@/_20pt/[rr]^{\varphi^k_\ell} \ar@/_30pt/[rrr]_{\varphi^i_j} & \cdots \ar[r] \ar[ur]^D & \Ckl \ar[r] \ar[ur]^D & \Cij \ar[r] \ar[ur]^D & \cdots\\
          }
        \end{displaymath}
        Composition of functions give a family of maps $D^k_\ell:\S\rightarrow\Ckl$ so that the diagram below commutes:
        \begin{displaymath}
          \xymatrix{
            \S \ar@/^20pt/[rr]_{\varphi^k_\ell} \ar@/^30pt/[rrr]^{\varphi^i_j} && \Ckl \ar[r] & \Cij \\
            \S \ar[urr] \ar[urrr] \\
          }
        \end{displaymath}
        These maps take Schwartz functions to their derivatives in some \Ckl and commute with the transitions $\Ckl\rightarrow\Cij$.
        Thus the map $D:\S\rightarrow\S$ induced by the limit definition takes functions to their derivatives, and is continuous.
      \end{proof}

      \begin{claim}
        \label{claim:translcont}
        For any $x\in\R$ the translation operator $T:\R\times\S\rightarrow\S$ given by $T(x,\varphi) = T_x\varphi$ is jointly continuous.
      \end{claim}
      \begin{proof}
        Let $Z$ be a topological space with continuous functions $f:Z\rightarrow\R$ and $g:Z\rightarrow\S$.
        By definition, $T$ is (jointly) continuous if and only if the map $h(z)=T_{f(z)}g(z)$ is continuous.

        Let $\{z_i\}\rightarrow z$ be a convergent sequence in $Z$, and consider the sequence $\{h(z_i)-h(z)\}$ in \S, which we hope to show approaches the zero function.
        Calculate
        \begin{align*}
          \left| (h(z_i)-h(z))(x) \right|
          &= \left| (T_{f(z_i)}g(z_i)-T_{f(z)}g(z))(x) \right|
          = \left| g(z_i)(x+f(z_i)) - g(z)(x+f(z)) \right| \text{.}
        \end{align*}
        Given some $\varepsilon>0$, for sufficiently large $i$ see that $|f(z_i)-f(z)|<\varepsilon$.
        For any fixed $i$ and some $a\in\R$, by the continuity of Schwartz functions there is a $\delta_a$ so that $|x-a|<\varepsilon$ implies $|g(z_i)(a+x)-g(z_i)(a)|<\delta_a$.
        Let $\delta=\sup_{a\in\R} \delta_a$, which is finite because each $\delta_a$ is finite, as it is the sup norm of a Schwartz function.
        As $\varepsilon\rightarrow0$ we see that $\delta$ approaches zero because each $\delta_a$ does.

        Thus for sufficiently large $i$ 
        \begin{align*}
          \left| (h(z_i)-h(z))(x) \right|
          &= \left| g(z_i)(x+f(z_i)) - g(z)(x+f(z)) \right|
          \\&\le \delta + \left| g(z_i)(x+f(z)) - g(z)(x+f(z)) \right|
          \\&\le \delta + \left|g(z_i)-g(z)\right|_\Ckl
        \end{align*}
        for any $k,\ell>0$.
        See that the right hand side becomes arbitrarily small as~$\varepsilon\rightarrow0$ and~$i\rightarrow\infty$, so the sequence $\{h(z_i)-h(z)\}$ approaches zero in \S.
        Thus the translation map $T$ is continuous.
      \end{proof}

      \begin{thm}
        \label{thm:sevalcts}
        For any $x\in\R$, the evaluation functional $\S\rightarrow\C$ given by $ \varphi\mapsto\varphi(x) $ is continuous.
      \end{thm}
      \begin{proof}
        Consider a sequence of Schwartz functions $\{\varphi_i\}\rightarrow0$.
        For any $k,\ell>1$, by the definition of projective limit the inclusion $\S\rightarrow\Ckl$ is continuous, so $\{\varphi_i\}\rightarrow 0$ in \Ckl.
        By \cref{claim:cklevalects} the evaluation-at-x functional on \Ckl is also continuous.
        Thus the composition is continuous and the diagram below is a commutative diagram of continuous maps:
        \begin{displaymath}
          \xymatrix{
          \S \ar[r]^{\text{incl.}} \ar[dr]_{\text{eval-at-}x} & \Ckl \ar[d]^{\text{eval-at-}x}\\
          & \C
          }
        \end{displaymath}
      \end{proof}

\end{document}
