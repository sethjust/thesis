\chapter{Fourier Inversion}

  Define $\psi:\R\times\R\rightarrow\C$ by $\psi(x,\xi) = e^{2\pi i \xi x}$.
  As $x$ varies for a fixed $\xi$, $\psi$ is the frequency-$\xi$ oscillation on the line -- a ``pure tone''.
  Just as interesting sounds are combinations of different frequencies, we can view functions as being built from these oscillations. \todo{sell this much better}

  Given a function $f:\R\rightarrow\C$, we can measure how much it ``matches'' the behavior of some frequency-$\xi$ oscillation.
  The idea is that by looking at what frequencies $f$ matches, we'll know exactly what oscillations it's built out of.
  The frequency spectrum or \emph{Fourier transform} of $f$ is a new function $\F f:\R\rightarrow\C$ defined by $\F f(\xi) = \int_{x\in\R} f(x)\overline{\psi(x,\xi)}dx$.

  In addition to finding the component frequencies of a function, we can combine together oscillations to create a new function with the \emph{inverse Fourier transform}:
  \begin{align*}
    \F^{-1}g&:\R\rightarrow\C\\
    \F^{-1}g(x) &= \int_{\xi\in\R} g(\xi)\psi(x,\xi)d\xi
  \end{align*}

  Intuitively we want to be able to say that $\F^{-1}g$ is a function that has the frequency spectrum described by $g$, or that $\F\F^{-1} g = g$, so the normal and inverse Fourier transforms are inverses.
  However, this claim is difficult to make because the integrals are not always defined -- for example $\F1(0)=\int_{-\infty}^\infty1$.
  In order to talk meaningfully about Fourier inversion we'll need an environment in which Fourier transforms are sensible -- that is, we need to limit our space of functions so that the Fourier transform works nicely.
  It is worth noting at this point that $\F^{-1}g(x) = \F g(-x)$, and so any properties of the Fourier transform extend to the inverse Fourier transform, meaning we only need consider the former.

  From Fourier analysis we know the following properties of the Fourier transform: \todo{where?}
  \begin{itemize}
    \item The smoother $f$ is the faster $\F f$ decays.
      If $f$ has integrable derivatives of order up to $k$, then $\F f(x)$ decays at least as quickly as $|x|^{-k}$ as $x$ approaches $\infty$.
    \item The faster $f$ decays the smoother $\F f$ is.
      If $f$ decays at least as quickly as $|x^{-k}|$ as $x\rightarrow\infty$, then $\F f$ has integrable derivatives of order up to $k-2$. % when dom f = \R^n this is k-n-1
  \end{itemize}
  We can use these facts to find functions that behave nicely with regard to the Fourier transform.
  \begin{defn}
    A \emph{Schwartz function} is a function
    \begin{equation*}
      \varphi:\R\rightarrow\C
    \end{equation*}
    that is \emph{smooth}, so for all $k\in\Z_{\ge0}$, the $k^\text{th}$ derivative $\varphi^{(k)}$ exists and is continuous.
    Furthermore, each derivative $\varphi^{(k)}$ decays faster than any polynomial: for any $k\in\Z_{\ge0}$ and $d\in\Z_{\ge1}$, $|\varphi^{(k)}(x)|\le|x|^{-d}$ for large enough $|x|$.
  \end{defn}
  Because of the decay condition on Schwartz functions their Fourier transform will \emph{always} be defined -- the integral will never diverge, so we can always talk about Fourier transforms of Schwartz functions.

  Using our facts about the Fourier transform we can see that for a Schwartz function $f$:
  \begin{itemize}
    \item $\F f$ decays faster than any power of $x$ because $f$ has infinitely many continuous derivatives.
    \item $\F f$ has infinitely many derivatives because $f$ decays faster than any power of $x$.
  \end{itemize}
  Thus $\F f$ is a Schwartz function, and so, because $f(-x)$ is also Schwartz, $\F^{-1}f$ is \emph{also} Schwartz.

  Now that we have an appropriate environment, we can properly investigate the relationship we want to find between the normal and inverse Fourier transforms.
  \begin{thm}
    \label{thm:fourinv}
    $\F^{-1}\F \varphi = \varphi \text{ for } \varphi\in\{\text{Schwartz Functions}\}$.
  \end{thm}
  To prove Theorem \ref{thm:fourinv} we first reduce it to the single case of $x=0$.
  Consider the following:
  \begin{proof}[Proof of Fourier inversion, granting inversion at 0]
    \todo{Introduce translation operators}
    \begin{align*}
      (\F^{-1}\F f)(x) 
    \end{align*}
    \todo{finish proof}
  \end{proof}

  \begin{rmk}
    The proof given here, while fairly simple, only applies to the 1-dimensional case. \todo{The proof for Euclidean space}
  \end{rmk}
